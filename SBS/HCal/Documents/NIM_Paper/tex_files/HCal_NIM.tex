\documentclass[review]{elsarticle}

\usepackage{lineno,hyperref}
\modulolinenumbers[5]
\usepackage{amsmath} %Use \text in math mode.
%Allows for the making of cells in tables.
\usepackage{makecell}

\journal{Journal of \LaTeX\ Templates}

\newcommand{\hcal}{HCAL-J }
\newcommand{\jlab}{Jefferson Lab }
\newcommand{\q}{$Q^2$ }

%%%%%%%%%%%%%%%%%%%%%%%
%% Elsevier bibliography styles
%%%%%%%%%%%%%%%%%%%%%%%
%% To change the style, put a % in front of the second line of the current style and
%% remove the % from the second line of the style you would like to use.
%%%%%%%%%%%%%%%%%%%%%%%

%% Numbered
%\bibliographystyle{model1-num-names}

%% Numbered without titles
%\bibliographystyle{model1a-num-names}

%% Harvard
%\bibliographystyle{model2-names.bst}\biboptions{authoryear}

%% Vancouver numbered
%\usepackage{numcompress}\bibliographystyle{model3-num-names}

%% Vancouver name/year
%\usepackage{numcompress}\bibliographystyle{model4-names}\biboptions{authoryear}

%% APA style
%\bibliographystyle{model5-names}\biboptions{authoryear}

%% AMA style
%\usepackage{numcompress}\bibliographystyle{model6-num-names}

%% `Elsevier LaTeX' style
\bibliographystyle{elsarticle-num}
%%%%%%%%%%%%%%%%%%%%%%%

\begin{document}

\begin{frontmatter}

\title{A Hadron Calorimeter for Nucleon Form Factor Measurements}
%\tnotetext[mytitlenote]{Fully documented templates are available in the elsarticle package on \href{http://www.ctan.org/tex-archive/macros/latex/contrib/elsarticle}{CTAN}.}

%% Group authors per affiliation:
\author[jlab]{S. Barcus\corref{mycorrespondingauthor}}
\cortext[mycorrespondingauthor]{Corresponding author}
\ead{skbarcus@jlab.org}
%\address{Radarweg 29, Amsterdam}
%\author[jlab]{S. Barcus\fnref{myfootnote}}
%\fntext[myfootnote]{Jefferson Lab.}

%% or include affiliations in footnotes:
%\author[mymainaddress,mysecondaryaddress]{Elsevier Inc}
%\ead[url]{www.elsevier.com}

%\author[jlab]{Jefferson Lab}
\address[jlab]{Jefferson Laboratory}

%\author[mysecondaryaddress]{Global Customer Service\corref{mycorrespondingauthor}}
%\ead{support@elsevier.com}

%\address[mymainaddress]{1600 John F Kennedy Boulevard, Philadelphia}
%\address[mysecondaryaddress]{360 Park Avenue South, New York}

\begin{abstract}
\hcal is a sampling calorimeter designed to measure the energy of several GeV protons and neutrons. It will be used for measuring hadron energy and triggering purposes in the upcoming Super BigBite Spectrometer (SBS) program to study nucleon form factors at the Thomas Jefferson National Accelerator Facility (Jefferson Lab). \hcal has an energy resolution of ***\% and a position resolution of *** cm. 
\end{abstract}

\begin{keyword}
calorimetry\sep nucleon form factors \sep \hcal
%\MSC[2010] 00-01\sep  99-00
\end{keyword}

\end{frontmatter}

\linenumbers

\section{Introduction}

\jlab uses the Continuous Electron Beam Accelerator Facility (CEBAF) to accelerate electrons for scattering experiments. After a recently completed upgrade CEBAF is capable of accelerating electrons up to 12 GeV. This increased maximum energy allows the lab to probe higher \q regions than previously available. One aspect of this 12 GeV era is the Super BigBite Spectrometer (SBS) program in Hall A. SBS consists of the SBS dipole magnet which curves the trajectory of scattered hadrons, the \hcal for energy measurements, Gas Electron Multipliers (GEMs) for particle tracking, polarimeters, a coordinate detector, and the refurbished BigBite detector package which includes a hodoscope, gas Cherenkov, and shower calorimeters \cite{brio_2018}. The SBS program will measure the nucleon form factors $G_M^n$ \cite{gmn_proposal}, $G_E^n$ \cite{gen_proposal}, and $G_E^p$ \cite{gep_proposal} at significantly higher \q than has been done before. \\

Since the initial measurement of the proton's electromagnetic form factors in the 1950's \cite{hofstadter_1956} great strides have been made in understanding the nucleon form factors. By the 1990's the nucleon form factors: $G_E^p$, $G_M^p$, $G_E^n$, and $G_M^n$ were found to generally follow a dipole description $F_{dipole} = \left( 1+\frac{Q^2}{0.71 \;GeV^2}\right)^{-2}$ \cite{bosted_1995}. The ratio of the proton's Sachs form factors, $R_p(Q^2) = \mu_{p}\frac{G_E^p}{G_M^p}$, was found to be approximately unity up to $Q^2$ $\approx$ 1 GeV. However, the nucleon form factor data up until the the mid 1990's all came from unpolarized Rosenbluth separations.\\

At the turn of the century physicists began measuring the nucleon form factors using the polarization transfer method first proposed in the late 1950's \cite{polarization_transfer} as well as the double polarization method proposed in the early 1980's \cite{double_polarization} ***(check correct paper). These polarization techniques are far less sensitive to two photon exchange effects than traditional Rosenbluth separations. Since the early 2000's many such polarization measurements of the nucleon form factors have been made in the high \q region of $1\; \text{GeV}^2 < Q^2 < 10\; \text{GeV}^2$. These new polarization measurements were in stark disagreement with form factor measurements from Rosenbluth techniques in this higher $Q^2$ region. Polarization methods indicated that the ratio of the proton's form factors, $R_p(Q^2)$, diverged from unity with the proton electric form factor falling off far more rapidly than the magnetic form factor as shown in Figure \ref{fig:polarization_vs_rosenbluth}. Whereas, the Rosenbluth results remained consistent with an $R_p(Q^2) \approx 1$, albeit with larger uncertainties than the polarized measurements.\\

	\begin{figure}[!ht]
	\begin{center}
	\includegraphics[width=0.6\linewidth]{/home/skbarcus/JLab/SBS/HCal/Documents/NIM_Paper/pictures/Rosenbluth_vs_Polarized_FFs_Clean.png}
	\end{center}
	\caption{
	{\bf{Ratio of Proton's Sachs Form Factors, $R_p(Q^2)$.}} The dotted line at unity indicates adherence to dipole form factors. The solid and dashed lines are fits to the data. The experiments listed in the top half of the image are unpolarized Rosenbluth separations, and the experiments in the lower half of the image used polarization techniques. Image from \cite{cisbani_2014}***need clearer image or maybe the ones from other paper.}
	\label{fig:polarization_vs_rosenbluth}
	\end{figure}	

The SBS nucleon form factor experiments will explore even higher \q regions than previously. The $G_M^n$ experiment with measure the magnetic form factor of the neutron by measuring quasielastic electron scattering cross section ratios d(e,e'n)p/d(e,e'p)n off of liquid deuterium \cite{gmn_proposal}. The $G_E^n$ experiment will measure the $G_E^n$/$G_M^n$ ratio using the double polarization technique to measure the asymmetry of electron scattering off of polarized $^3$He \cite{gen_proposal}. Using the high-precision $G_M^n$ data acquired by SBS $G_E^n$ can be directly extracted from this ratio. The $G_E^p$ experiment will measure the ratio $G_E^p$/$G_M^p$ at high $Q^2$ using the polarization transfer method to scatter electrons from a liquid(***check) hydrogen target \cite{gep_proposal}. Table \ref{tab:q2_ranges} summarizes the \q ranges to be measured by the SBS nucleon form factor experiments. 

 	\begin{table}[h]
	\centering
	%\caption{Spectrometer Central Kinematics}%Prints title above table.
	\begin{tabular}{|cc|}
	\hline
	\makecell{Nucleon\\ Form Factor} & \makecell{\q Range\\ $[$GeV$^2]$}\\
	\hline
	$G_E^p$ & 5.0-12.0\\
    $G_M^p$ & 4.8-14.0\\
    $G_E^n$ & 1.5-10.2\\
    $G_M^n$ & 3.5-13.5\\
	\hline
	\end{tabular}
	%\label{tab:edep}
	\caption{{\bf{Kinematic Ranges of the SBS Nucleon Form Factor Experiments.}} ***check ranges.} %Caption* supresses printing of second caption saying Table number again.
	\label{tab:q2_ranges}
	\end{table}
	
	Precision measurements of the nucleon form factors will improve uncertainties at lower $Q^2$, while at high \q these measurements will offer important new physics insights. These measurements are a rigorous test for competing lattice QCD, pQCD, VMD models, and effective field theory predictions which have large disagreements in the high \q region (*** add plot showing theory curves?). This new data will also increase the \q range of the form factor flavor decompositions. Additionally, because the nucleon form factors $F_1$ and $F_2$ equal the first moments of the $H^q$ and $E^q$ generalized parton distribution functions (GPDs), the nucleon form factors provide an important constraint for developing GPD models \cite{gen_proposal}. The $H^q$ and $E^q$ GPDs are directly related to quark orbital angular momentum \cite{cisbani_2014}(***maybe use his ref 5) meaning nucleon form factor measurements will provide tantalizing new insight into this quantity.  

\section{Hadron Calorimeter}

\hcal consists of 288 individual modules arranged in 12 columns and 24 rows as shown in Figure \ref{fig:HCal}. These modules are spread across four craneable subassemblies, and the detector weighs approximately 40 tons in total. Each module is made up of 40 layers of 1 cm thick scintillator (PPO only, 2,5-Diphenyloxazole) alternating with 40 layers of 1.5 cm thick iron absorbers as shown in Figure \ref{fig:HCal_interior}, and each module measures 15$\times$15 cm$^2$ with a length of 1 m. The hadrons strike the iron causing them to shower, and the scintillators produce photons from these shower particles. In the center of the iron and scintillators is a St. Gobain BC-484 wavelength shifter (decay time 3 ns) which improves light collection efficiency and uniformity \cite{brio_2018}. The photons in the wavelength shifter are transported to photomultiplier tubes (PMTs) on one end of the modules via custom built light guides that can be seen in the lower image of Figure \ref{fig:HCal_interior}.

	\begin{figure}[!ht]
	\begin{center}
	\includegraphics[width=0.4\linewidth]{/home/skbarcus/Documents/JLab_SS1/Seminar/HCal_External_Clean.png}
	\end{center}
	\caption{
	{\bf{SBS Hadron Calorimeter.}} \hcal is composed of 288 PMT modules divided into four separate subassemblies which can be moved by crane (total weight $\approx$40 tons). The fully assembled HCal will have 12 columns and 24 rows of modules with PMTs attached. Image from \cite{brio_2018}.}
	\label{fig:HCal}
	\end{figure}	
	
	\begin{figure}[!ht]
	\begin{center}
	\includegraphics[width=0.65\linewidth]{/home/skbarcus/Documents/JLab_SS1/Seminar/HCal_Interior_Clean.png}
	\includegraphics[width=0.85\linewidth]{/home/skbarcus/JLab/SBS/HCal/Documents/NIM_Paper/pictures/HCal_Interior_Light_Guide_Clean.png}
	\end{center}
	\caption{
	{\bf{SBS Hadron Calorimeter Module Interior.}} The interior of each \hcal module is comprised of alternating layers of iron absorbers and scintillators. The hadrons shower in the iron and then these showers create photons in the scintillators. These photons pass through a wavelength shifter before being transported into the PMTs via light guides. Image from \cite{brio_2018}.}
	\label{fig:HCal_interior}
	\end{figure}	
	

\section{High Voltage System}

\section{Data Acquisition}

\section{\hcal Performance}

\section{Summary}

\section*{References}

\bibliography{mybibfile}

\end{document}