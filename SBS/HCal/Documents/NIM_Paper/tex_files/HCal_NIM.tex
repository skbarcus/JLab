\documentclass[review]{elsarticle}

\usepackage{lineno,hyperref}
\modulolinenumbers[5]
\usepackage{amsmath} %Use \text in math mode.
%Allows for the making of cells in tables.
\usepackage{makecell}

\journal{Journal of \LaTeX\ Templates}

\newcommand{\hcal}{HCAL-J }
\newcommand{\jlab}{Jefferson Lab }
\newcommand{\q}{$Q^2$ }

%%%%%%%%%%%%%%%%%%%%%%%
%% Elsevier bibliography styles
%%%%%%%%%%%%%%%%%%%%%%%
%% To change the style, put a % in front of the second line of the current style and
%% remove the % from the second line of the style you would like to use.
%%%%%%%%%%%%%%%%%%%%%%%

%% Numbered
%\bibliographystyle{model1-num-names}

%% Numbered without titles
%\bibliographystyle{model1a-num-names}

%% Harvard
%\bibliographystyle{model2-names.bst}\biboptions{authoryear}

%% Vancouver numbered
%\usepackage{numcompress}\bibliographystyle{model3-num-names}

%% Vancouver name/year
%\usepackage{numcompress}\bibliographystyle{model4-names}\biboptions{authoryear}

%% APA style
%\bibliographystyle{model5-names}\biboptions{authoryear}

%% AMA style
%\usepackage{numcompress}\bibliographystyle{model6-num-names}

%% `Elsevier LaTeX' style
\bibliographystyle{elsarticle-num}
%%%%%%%%%%%%%%%%%%%%%%%

\begin{document}

\begin{frontmatter}

\title{A Hadron Calorimeter for Nucleon Form Factor Measurements}
%\tnotetext[mytitlenote]{Fully documented templates are available in the elsarticle package on \href{http://www.ctan.org/tex-archive/macros/latex/contrib/elsarticle}{CTAN}.}

%% Group authors per affiliation:
\author[jlab]{S. Barcus\corref{mycorrespondingauthor}}
\cortext[mycorrespondingauthor]{Corresponding author}
\ead{skbarcus@jlab.org}
%\address{Radarweg 29, Amsterdam}
%\author[jlab]{S. Barcus\fnref{myfootnote}}
%\fntext[myfootnote]{Jefferson Lab.}

%% or include affiliations in footnotes:
%\author[mymainaddress,mysecondaryaddress]{Elsevier Inc}
%\ead[url]{www.elsevier.com}

%\author[jlab]{Jefferson Lab}
\address[jlab]{Jefferson Laboratory}

%\author[mysecondaryaddress]{Global Customer Service\corref{mycorrespondingauthor}}
%\ead{support@elsevier.com}

%\address[mymainaddress]{1600 John F Kennedy Boulevard, Philadelphia}
%\address[mysecondaryaddress]{360 Park Avenue South, New York}

\begin{abstract}
\hcal is a sampling calorimeter designed to measure the energy of several GeV protons and neutrons. It will be used for measuring hadron energy and triggering purposes in the upcoming Super BigBite Spectrometer (SBS) program to study nucleon form factors at the Thomas Jefferson National Accelerator Facility (Jefferson Lab). \hcal has an energy resolution of ***\% and a position resolution of *** cm. 
\end{abstract}

\begin{keyword}
calorimetry\sep nucleon form factors \sep \hcal
%\MSC[2010] 00-01\sep  99-00
\end{keyword}

\end{frontmatter}

\linenumbers

\section{Introduction}
\label{intro}

\jlab uses the Continuous Electron Beam Accelerator Facility (CEBAF) to accelerate electrons for scattering experiments. After a recently completed upgrade CEBAF is capable of accelerating electrons up to 12 GeV. This increased maximum energy allows the lab to probe higher \q regions than previously available. One aspect of this 12 GeV era is the Super BigBite Spectrometer (SBS) program in Hall A. SBS consists of the SBS dipole magnet which curves the trajectory of scattered hadrons, the \hcal for energy measurements, Gas Electron Multipliers (GEMs) for particle tracking, polarimeters, a coordinate detector, and the refurbished BigBite detector package which includes a hodoscope, gas Cherenkov, and shower calorimeters \cite{brio_2018}. The SBS program will measure the nucleon form factors $G_M^n$ \cite{gmn_proposal}, $G_E^n$ \cite{gen_proposal}, and $G_E^p$ \cite{gep_proposal} at significantly higher \q than has been done before. \\

Since the initial measurement of the proton's electromagnetic form factors in the 1950's \cite{hofstadter_1956} great strides have been made in understanding the nucleon form factors. By the 1990's the nucleon form factors: $G_E^p$, $G_M^p$, $G_E^n$, and $G_M^n$ were found to generally follow a dipole description $F_{dipole} = \left( 1+\frac{Q^2}{0.71 \;GeV^2}\right)^{-2}$ \cite{bosted_1995}. The ratio of the proton's Sachs form factors, $R_p(Q^2) = \mu_{p}\frac{G_E^p}{G_M^p}$, was found to be approximately unity up to $Q^2$ $\approx$ 1 GeV. However, the nucleon form factor data up until the the mid 1990's all came from unpolarized Rosenbluth separations.\\

At the turn of the century physicists began measuring the nucleon form factors using the polarization transfer method first proposed in the late 1950's \cite{polarization_transfer} as well as the double polarization method proposed in the early 1980's \cite{double_polarization} ***(check correct paper). These polarization techniques are far less sensitive to two photon exchange effects than traditional Rosenbluth separations. Since the early 2000's many such polarization measurements of the nucleon form factors have been made in the high \q region of $1\; \text{GeV}^2 < Q^2 < 10\; \text{GeV}^2$. These new polarization measurements were in stark disagreement with form factor measurements from Rosenbluth techniques in this higher $Q^2$ region. Polarization methods indicated that the ratio of the proton's form factors, $R_p(Q^2)$, diverged from unity with the proton electric form factor falling off far more rapidly than the magnetic form factor as shown in Figure \ref{fig:polarization_vs_rosenbluth}. Whereas, the Rosenbluth results remained consistent with an $R_p(Q^2) \approx 1$, albeit with larger uncertainties than the polarized measurements.\\

	\begin{figure}[!ht]
	\begin{center}
	\includegraphics[width=0.6\linewidth]{/home/skbarcus/JLab/SBS/HCal/Documents/NIM_Paper/pictures/Rosenbluth_vs_Polarized_FFs_Clean.png}
	\end{center}
	\caption{
	{\bf{Ratio of Proton's Sachs Form Factors, $R_p(Q^2)$.}} The dotted line at unity indicates adherence to dipole form factors. The solid and dashed lines are fits to the data. The experiments listed in the top half of the image are unpolarized Rosenbluth separations, and the experiments in the lower half of the image used polarization techniques. Image from \cite{cisbani_2014}***need clearer image or maybe the ones from other paper.}
	\label{fig:polarization_vs_rosenbluth}
	\end{figure}	

The SBS nucleon form factor experiments will explore even higher \q regions than previously. The $G_M^n$ experiment with measure the magnetic form factor of the neutron by measuring quasielastic electron scattering cross section ratios d(e,e'n)p/d(e,e'p)n off of liquid deuterium \cite{gmn_proposal}. The $G_E^n$ experiment will measure the $G_E^n$/$G_M^n$ ratio using the double polarization technique to measure the asymmetry of electron scattering off of polarized $^3$He \cite{gen_proposal}. Using the high-precision $G_M^n$ data acquired by SBS $G_E^n$ can be directly extracted from this ratio. The $G_E^p$ experiment will measure the ratio $G_E^p$/$G_M^p$ at high $Q^2$ using the polarization transfer method to scatter electrons from a liquid(***check) hydrogen target \cite{gep_proposal}. Table \ref{tab:q2_ranges} summarizes the \q ranges to be measured by the SBS nucleon form factor experiments.\\ 

 	\begin{table}[h]
	\centering
	%\caption{Spectrometer Central Kinematics}%Prints title above table.
	\begin{tabular}{|cc|}
	\hline
	\makecell{Nucleon\\ Form Factor} & \makecell{\q Range\\ $[$GeV$^2]$}\\
	\hline
	$G_E^p$ & 5.0-12.0\\
    $G_M^p$ & 4.8-14.0\\
    $G_E^n$ & 1.5-10.2\\
    $G_M^n$ & 3.5-13.5\\
	\hline
	\end{tabular}
	%\label{tab:edep}
	\caption{{\bf{Kinematic Ranges of the SBS Nucleon Form Factor Experiments.}} ***check ranges.} %Caption* supresses printing of second caption saying Table number again.
	\label{tab:q2_ranges}
	\end{table}
	
	Precision measurements of the nucleon form factors will improve uncertainties at lower $Q^2$, while at high \q these measurements will offer important new physics insights. These measurements are a rigorous test for competing lattice QCD, pQCD, VMD models, and effective field theory predictions which have large disagreements in the high \q region (*** add plot showing theory curves?). This new data will also increase the \q range of the form factor flavor decompositions. Additionally, because the nucleon form factors $F_1$ and $F_2$ equal the first moments of the $H^q$ and $E^q$ generalized parton distribution functions (GPDs), the nucleon form factors provide an important constraint for developing GPD models \cite{gen_proposal}. The $H^q$ and $E^q$ GPDs are directly related to quark orbital angular momentum \cite{cisbani_2014}(***maybe use his ref 5) meaning nucleon form factor measurements will provide tantalizing new insight into this quantity. \\ 

\section{Hadron Calorimeter}
\label{hcal}

\hcal consists of 288 individual modules arranged in 12 columns and 24 rows as shown in Figure \ref{fig:HCal}. These modules are spread across four craneable subassemblies, and the detector weighs approximately 40 tons in total. Each module is made up of 40 layers of 1 cm thick scintillator (PPO only, 2,5-Diphenyloxazole) alternating with 40 layers of 1.5 cm thick iron absorbers as shown in Figure \ref{fig:HCal_interior}, and each module measures 15$\times$15 cm$^2$ with a length of 1 m. The hadrons strike the iron causing them to shower, and the scintillators produce photons from these shower particles. In the center of the iron and scintillators is a St. Gobain BC-484 wavelength shifter (decay time 3 ns) which improves light collection efficiency and uniformity \cite{brio_2018}. The photons in the wavelength shifter are transported to photomultiplier tubes (PMTs) on one end of the modules via custom built light guides that can be seen in the lower image of Figure \ref{fig:HCal_interior}.\\

	\begin{figure}[!ht]
	\begin{center}
	\includegraphics[width=0.4\linewidth]{/home/skbarcus/Documents/JLab_SS1/Seminar/HCal_External_Clean.png}
	\end{center}
	\caption{
	{\bf{SBS Hadron Calorimeter.}} \hcal is composed of 288 PMT modules divided into four separate subassemblies which can be moved by crane (total weight $\approx$40 tons). The fully assembled HCal will have 12 columns and 24 rows of modules with PMTs attached. Image from \cite{brio_2018}.}
	\label{fig:HCal}
	\end{figure}	
	
	\begin{figure}[!ht]
	\begin{center}
	\includegraphics[width=0.65\linewidth]{/home/skbarcus/Documents/JLab_SS1/Seminar/HCal_Interior_Clean.png}
	\includegraphics[width=0.85\linewidth]{/home/skbarcus/JLab/SBS/HCal/Documents/NIM_Paper/pictures/HCal_Interior_Light_Guide_Clean.png}
	\end{center}
	\caption{
	{\bf{SBS Hadron Calorimeter Module Interior.}} The interior of each \hcal module is comprised of alternating layers of iron absorbers and scintillators. The hadrons shower in the iron and then these showers create photons in the scintillators. These photons pass through a wavelength shifter before being transported into the PMTs via light guides. Image from \cite{brio_2018}.}
	\label{fig:HCal_interior}
	\end{figure}	

\subsection{Calorimeter Modules}
\label{modules}

\subsection{Cabling System}
\label{cables}

The cabling scheme for the \hcal is designed such that all detector channels can be accessed at numerous points between the detector PMTs and where the final signals enter the DAQ electronics. The \hcal cable system can be broken into three groups based on whether the physics signals flow to the fADC250s or the F1TDCs discussed in Section \ref{daq} or the summing modules discussed in Section \ref{electronics}. Five racks hold the front-end and DAQ cabling and electronics and a diagrammatic layout of these racks is show in Figures \ref{fig:fe} and \ref{fig:daq}. The front-end racks are labelled RR1, RR2, and RR3. The electronics and cables in RR1 and RR3 are mirrored as each of these racks contains half of the \hcal channels. The DAQ side racks are RR4 and RR5. The DAQ side is connected to the front-end via 100 m*** long BNC cables running from RR2 to RR4. \\ 

	\begin{figure}[!ht]
	\begin{center}
	\includegraphics[width=0.8\linewidth]{/home/skbarcus/JLab/SBS/HCal/Schematics/My_Maps/HCal_FE.png}
	\end{center}
	\caption{
	{\bf{\hcal Front-End Layout.}} The front-end consists of racks RR1, RR2, and RR3. Signal enters the front-end through the amplifiers at the bottoms of racks RR1 and RR3. RR1 and RR3 are mirrored and each handle half of the 288 \hcal channels. The front-end is connected to the DAQ side via 100 m*** long BNC cables connecting RR2 to RR4.}
	\label{fig:fe}
	\end{figure}	
	
	\begin{figure}[!ht]
	\begin{center}
	\includegraphics[width=0.8\linewidth]{/home/skbarcus/JLab/SBS/HCal/Schematics/My_Maps/HCal_DAQ.png}
	\end{center}
	\caption{
	{\bf{\hcal DAQ Layout.}} The DAQ side consists of racks RR4 and RR5. Rack RR5 contains the fADC250s and F1TDCs into which the physics signal flows. The DAQ side is connected to the front-end via 100 m*** long BNC cables connecting RR4 to RR2.}
	\label{fig:daq}
	\end{figure}	

Beginning at the detector PMTs the physics signal can be traced to the fADC250s, from which both energy and timing measurements can be made. The analog signal exits the detector PMTs and enters the Phillips Scientific 776 amplifiers at the base of RR1 and RR3 depending on from which half of the detector the signal originated. The PS776 amplifiers have dual outputs which each produce a 10$\times$ amplified analog signal. From one of these outputs the amplified physics signal flows to patch panels in the bottom of RR2 which connect to DAQ side patch panels at the base of RR4 via 100 m*** long BNC cables. Once emerging from RR4 on the DAQ side the signals flow into the fADC250s in RR5 and are recorded for analysis. The following list gives each component dedicated to processing the fADC signals in the order in which the signal passes through them:\\

\begin{itemize}\itemsep6pt \parskip0pt \parsep0pt
	\item 288 detector PMTs (192 of the PMTs are 12 stage 2'' Photonis XP2262 PMTs and 96 are 8 stage stage 2'' Photonis XP2282 PMTs).
	\item 288 5 m***check BNC-LEMO cables. 
	\item 18 PS776 dual output 10$\times$ amplifiers in RR1 and RR3.
	\item 288 2 m LEMO-BNC cables. 
	\item 5 BNC-BNC patch panels in RR2. 
	\item 288 100 m*** BNC-BNC*** cables. 
	\item 5 BNC-BNC*** patch panels in RR4. 
	\item 288 2 m BNC-LEMO cables. 
	\item 18 fADC250s in RR5.
%	\item 288 5 m***check BNC-LEMO cables. Connect detector PMTs to amplifiers in RR1 and RR3.
%	\item 18 PS776 dual output 10$\times$ amplifiers. Output signal to 
%	\item 288 2 m LEMO-BNC cables. Connect the first amplifier outputs in RR1 and RR3 to patch panels in RR2.
%	\item 5 BNC-BNC patch panels (RR2). Patch first amplifier outputs to 100 m* BNC cables.
%	\item 288 100 m*** BNC-BNC*** cables. Connect front-end patch panels in RR2 to DAQ side patch panels in RR4.
%	\item 5 BNC-BNC*** patch panels (RR4). Patch signal from long BNC cables to cables running to fADC250s.
%	\item 288 2 m BNC-LEMO cables. Connect patch panels in RR4 to fADC250s in RR5.
\end{itemize}

Timing information is derived from F1TDCs and flows through the front-end and DAQ side electronics in the following manner. An analog signal first exits the detector PMTs and flows into the Phillips Scientific 776 10$\times$ amplifiers at the base of RR1 and RR3 depending on from which half of the detector the signal originated. Exiting the second of the two PS776 outputs the amplified analog signal travels to a 50-50 splitter panel with two sets of outputs. The halved signal then exits the first set of these outputs and travels to Phillips Scientific 706 discriminators with low ($\approx 11 mV$) thresholds in RR2. This now NIM logic signal passes into patch panels in RR2 and then over 100 m*** long BNC-BNC*** cables which connect to patch panels in RR4. After leaving the patch panels the physics signals enter a second set of LeCroy 2313 discriminators which ensure the signal shape continues to have a sharp leading edge. The second set of discriminators translate the signal into an ECL signal which then flows into the F1TDCs to be recorded. The following list gives each component dedicated to processing the F1TDC signals in the order in which the signal passes through them:\\

\begin{itemize}\itemsep6pt \parskip0pt \parsep0pt
	\item 288 detector PMTs (192 of the PMTs are 12 stage 2'' Photonis XP2262 PMTs and 96 are 8 stage stage 2'' Photonis XP2282 PMTs).
	\item 288 5 m***check BNC-LEMO cables. 
	\item 18 PS776 dual output 10$\times$ amplifiers in RR1 and RR3.
	\item 288 2 m LEMO-BNC cables. 
	\item 9 50-50 dual output splitter panels in RR1 and RR3. 
	\item 288 2 m BNC-LEMO cables.
	\item 18 PS706 discriminators in RR2.
	\item 288 2 m LEMO-BNC cables.
	\item 5 BNC-BNC patch panels in RR2.
	\item 288 100 m*** BNC-BNC*** cables. 
	\item 5 BNC-BNC*** patch panels in RR4. 
	\item 288 2m BNC cables.
	\item 18 Lecroy 2313 discriminators in RR4.
	\item 18 16 channel ribbon cables.
	\item 5 F1TDCs in RR5. 
%	\item 288 5 m***check BNC-LEMO cables. Connect detector PMTs to dual output front-end amplifiers in RR1 and RR3.
%	\item 288 2 m LEMO-BNC cables. Connect the second amplifier outputs in RR1 and RR3 to splitter panels in RR1 and RR3.
%	\item 9 50-50 dual output splitter panels in RR1 and RR3. Divide the signal in half with one set of outputs going to 2m BNC-LEMO cables.
%	\item 288 2 m BNC-LEMO cables. Connect one set of splitter outputs to PS706 discriminators.
%	\item 288 100 m*** BNC-BNC*** cables. Connect front-end patch panels in RR2 to DAQ side patch panels in RR4.
\end{itemize}

The third set of signals leading to the summing modules takes the following path. An analog signal first exits the detector PMTs and flows into the Phillips Scientific 776 10$\times$ amplifiers at the base of RR1 and RR3 depending on from which half of the detector the signal originated. Exiting the second of the two PS776 outputs the amplified analog signal travels to a 50-50 splitter panel with two sets of outputs. The halved signal then exits the second set of these outputs and travels to to the summing modules which sum the analog signal of 16 PMTs for triggering and analysis purposes. The following list gives each component dedicated to processing the summing module signals in the order in which the signal passes through them:\\

\begin{itemize}\itemsep6pt \parskip0pt \parsep0pt
	\item 288 detector PMTs (192 of the PMTs are 12 stage 2'' Photonis XP2262 PMTs and 96 are 8 stage stage 2'' Photonis XP2282 PMTs).
	\item 288 5 m***check BNC-LEMO cables. 
	\item 18 PS776 dual output 10$\times$ amplifiers in RR1 and RR3.
	\item 288 2 m LEMO-BNC cables. 
	\item 9 50-50 dual output splitter panels in RR1 and RR3. 
	\item 288 2 m BNC-LEMO cables.
	\item 18 summing modules***(better name? UVA?) in RR1 and RR3.
\end{itemize}

\subsection{Electronics}
\label{electronics}

\subsection{Triggers}
\label{triggers}

When particles strike \hcal and create showers these showers spread over multiple modules, generally around 3-7***, based on the energy of the incident particle. To reconstruct the energy of the particle the signals from each individual module that measured any energy being deposited must be summed together to find the particle's energy upon striking the detector. The summing modules*** sum the output signals of 4$\times$4 blocks of PMT signals which are each attached to one of the 288 individual detector modules. These 4$\times$4 blocks of 16 modules are then further summed as 8$\times$8 blocks of 64 modules, made up of four adjacent 4$\times$4 blocks of modules, as seen in Figure \ref{fig:sum_trigger}. Any combination of the sums of 16 blocks and/or the sums of 64 blocks can be used to create a trigger from the detector. The \hcal sum trigger will form a coincidence with the BigBite electron detector to create the main experimental trigger for the SBS nucleon form factor experiments.***check if HCal in all main triggers and what specific sums we plan to use.

 	\begin{figure}[!ht]
	\begin{center}
	\includegraphics[width=0.5\linewidth]{/home/skbarcus/Documents/JLab_SS1/Seminar/Summing_Module_Triggers.png}
	\end{center}
	\caption{
	{\bf{\hcal Module Sum Trigger.}} The diagram shows the layout of \hcal with its structure of 12$\times$24 individual detector modules as the small black squares. The summing modules*** sum 4$\times$4 blocks of individual detector module PMT signals. The 18 summed 4$\times$4 blocks are shown as red squares. These groups of 16 modules are then used to create larger sums of 64 individual detector module PMT signals in the form of four adjacent 4$\times$4 blocks of modules. The locations of these 10 sums of 64 modules, organized as 8$\times$8 blocks, are shown by blue circles at their centers surrounding the corners of the four 4$\times$4 blocks they sum together. Note that some 16 block sums are shared by more than one of the 64 block sums.}
	\label{fig:sum_trigger}
	\end{figure}	

A cosmic ray trigger is implemented with \hcal as well. This trigger consists of a plastic scintillator placed on top of the detector. When particles from cosmic rays pass through this scintillator a trigger is formed and each of the individual detector modules' fADCs and TDCs each read out. Cosmic rays are useful for calibrating detector settings like the high voltage settings as they are a constant presence with a consistent spectrum that can be measured when the electron beam is unavailable. Cosmic rays are also an excellent source to measure when testing the various electronics of the front-end and DAQ in preparation for upcoming experiments.

\subsubsection{LED System}
\label{leds}

\hcal is also equipped with an LED pulser system that can be used for triggering purposes. 

\subsection{Dry Air Supply}
\label{dry_air}

\section{High Voltage System}
\label{hv}

\section{Data Acquisition}
\label{daq}

\hcal integrates with the standard \jlab Hall A data acquisition system \cite{hall_a}. A core part of this setup is the CEBAF On-line Data Acquisition System (CODA) created by the \jlab data acquisition group. CODA runs on VME and VXS crates containing the individual DAQ modules using the VxWorks operation system as well as on linux workstations. Read-out controllers (ROCs) are a primary feature of CODA that run on the crates containing the DAQ modules. The ROCs hold the software that controls the individual DAQ modules while also taking the experimental data from the modules and holding it in a memory buffer. This buffered data is then transmitted to a workstation via network connection where the Event Builder (EB) arranges the data sent from the ROCs into event groupings. The EB then sends these events to the Event Recorder (ER) which writes the events to the local disk. The data is then transferred to tapes in the \jlab Mass Storage System (MSS). An Event Tranfser (ET) system is used to insert outside data into the event files throughout this process. This typically includes data such as scaler measurements, beam characteristics, and target conditions. CODA is controlled via the RunControl process which allows users to start and stop data runs, reconfigure the DAQ setup, and monitor the status of DAQ components via a GUI interface \cite{coda}.

The DAQ of \hcal is spread over two VXS crates, connected by an optical fiber, containing the individual ADC and TDC modules. Each crate contains a crate CPU, ROC, to control the individual DAQ modules and communicate through CODA with the Linux workstation. Each crate also contains a Trigger Interface (TI) module which creates a global clock for synchronization purposes as well as creates a trigger signal to control when data is recorded. This trigger signal is distributed by Signal Distributor (SD) modules which take the trigger signal from the TI and pass it along to each of the individual DAQ modules. The crate sharing fADC250s and F1TDC modules has two SD modules, and the crate with only the fADC250s contains one SD module. Each crate also contains one VXS Trigger Processor (VTP) module which contains a field programmable gate array (FPGA). This FPGA can be programmed with complex trigger logic algorithms to control the fADC250s.

ADC information for \hcal is obtained via JLab fADC250 modules. These flash ADC modules take numerous samples of the pulse signal when triggered at a rate of 250 MHz (4 ns sample sizes). The number of samples recorded is programmable along with a programmable latency. The fADC250s have an adjustable dynamic range of 0.5 V, 1 V, and 2 V which are selected using jumpers on the board itself. 18 fADC250s read in the 288 individual detector module signals via 16 LEMO inputs each on the front panel. The fADC250s are used to measure how much energy was deposited in a given physics signal, but they can also be used for timing information using a time over threshold technique. ***discuss timing etc in results section. ***probably want something on average pulse width and final window width and maybe latency. ***Probably want refs for at least fADC250s and F1TDCs.

Additional timing information for \hcal is measured by \jlab F1TDC modules which utilize an Acam-messelectronic F1 chip. Each F1TDC module contains eight F1 chips able to measure eight input channels with 120 ps timing resolution (64 channels total) or four input channels with 60 ps timing resolution (32 channels total). \hcal uses the 120 ps timing resolution mode and thus uses five F1TDCs to record all 288 detector module signals. The F1TDCs feature both programmable timing windows and latencies for observing the signals from the detector modules. Data is read in from 16 channel ECL ribbon cables on the front panel.  ***add final configuration?

\section{\hcal Performance}
\label{results}

\section{Summary}
\label{summary}

\section*{References}
\label{refs}

\bibliography{mybibfile}

\end{document}