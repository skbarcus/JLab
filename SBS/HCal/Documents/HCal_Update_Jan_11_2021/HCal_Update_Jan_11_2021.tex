\documentclass[10pt]{beamer}

%Allow captions for subfigures
\usepackage{subcaption}

%Setup for bibtex.
\usepackage[%
  sorting=none,%
%  backend=bibtex,%
  backend=biber,%
%  style=numeric,%
  style=phys,% 
  pageranges=false,% Print only first page, only works with style=phys
  chaptertitle=false,% for incollections show chapter titles - true for AIP style, false for APS style
  articletitle=true,% Print article title, AIP = false, APS = true
  biblabel=brackets,% Number entries by bracked notation
  related=true,%
  isbn=false,% Don't print ISBN
  doi=false,% Don't print DOI
  url=false,% Don't print URL
  eprint=false,% Don't print eprint information
  hyperref=true,%
%  note=false,%
  firstinits=true,% Use first intials only
  maxnames=3,% Truncate after N names
  minnames=1,% Print at least N names if truncated
%  natbib=true%
 ]{biblatex}
 
%Italicize et al.
 \DefineBibliographyStrings{english}{%
  andothers = {\textit{et al}\adddot}
}

%\addbibresource{bib_Books.bib}

%\begin{filecontents}{\jobname.bib}
%@article{Baird2002,
%author = {Baird, Kevin M and Hoffmann, Errol R and Drury, Colin G},
%journal = {Applied ergonomics},
%month = jan,
%number = {1},
%pages = {9--14},
%title = {{The effects of probe length on Fitts' law.}},
%volume = {33},
%year = {2002}
%}
%\end{filecontents}

\usetheme{metropolis}
\usepackage{appendixnumberbeamer}

\usepackage{booktabs}
\usepackage[scale=2]{ccicons}

\usepackage{pgfplots}
\usepgfplotslibrary{dateplot}

\usepackage{xspace}
\newcommand{\themename}{\textbf{\textsc{metropolis}}\xspace}

%Allows for the making of cells in tables.
\usepackage{makecell}

\usepackage{amsmath}
%\setbeamertemplate{itemize item}[circle]
%\setbeamertemplate{itemize subitem}[-]

\title{HCal Update}
\subtitle{}
\date{January 11\textsuperscript{th} 2020}
\author{The HCal Working Group}
\institute{Jefferson Lab}
% \titlegraphic{\hfill\includegraphics[height=1.5cm]{logo.pdf}}

\newcommand{\hcal}{HCal}

\begin{document}

\maketitle

\begin{frame}{Whole Detector Running}

	\begin{itemize}
		\item Previously running HCal as two independent halves.
		\item Now running whole detector at once. 
			\begin{itemize}
				\item[--] Trigger is OR of cosmic paddles above either detector half.
				\item[--] All fADCs set to 2 V range. $\rightarrow$ May have resolved DMA data transfer error!
				\item[--] DB updated to match final configuration of HCal stacked in hall. 
			\end{itemize}
	\end{itemize}

	\vspace{-3mm}
	\begin{center}
	    \includegraphics[width=1\linewidth]{/home/skbarcus/JLab/SBS/HCal/Pictures/Full_Detector/Both_Halves.jpg}
  	\end{center}

\end{frame}

\begin{frame}{Misc. Work}

	\begin{itemize}
		\item Tracked down several broken channels on the right half and fixed them.
		\item Swapped out a buggy HV card that didn't always give voltage readback data.
		\item Cable swap between HCal and BB worked out with Mark Jones.
			\begin{itemize}
				\item[--] HCal currently has 50 m cables for TDC, but needs $\approx$300 100 m cables.
				\item[--] HCal will swap it's 50 m TDC cables with BB's shower/preshower 100 m cables plus some ECAL cables.
				\item[--] Still need final HCal trigger cable. (100 m or less)
			\end{itemize} 
		\item Anticipate moving HCal to hall in April.
		\item HCal timing diagram created for coordinating with BB trigger.
	\end{itemize}

\end{frame}

\begin{frame}{HCal Timing Diagram}

	\begin{center}
	    \includegraphics[width=1\linewidth]{/home/skbarcus/JLab/SBS/HCal/Schematics/My_Maps/HCal_Timing_Diagram_Clean.png}
  	\end{center}

\end{frame}

\begin{frame}{LED System Work}

    \begin{itemize}
		\item HCal has 288 PMT modules. 
		\item Each module can observe 6 different LEDs over fibers.
		\item Each successive LED is roughly twice as bright as the previous LED.
        \item Powered by 8 LED power boxes on front-end (4 completed).
        \item Each LED power box controls 2 LED boxes on the sides of HCal (all 16 completed).
    \end{itemize}
    
    \begin{columns}[T,onlytextwidth]
	\column{0.5\textwidth}
	
	%For some incomprehensible reason these two images rotate when downloaded. To compensate they're rotated oddly in this online version.
	\vspace{-3mm}
	\begin{center}
	    \textbf{LED Power Boxes}
	    \includegraphics[width=0.8\textwidth, height=.45\textheight,angle=270]{/home/skbarcus/JLab/SBS/HCal/Documents/HCal_Update_Nov_9_2020/Pictures/LED_Power_Boxes_2.jpg}
  		%\includegraphics[width=3cm, height=4cm]{Images/LED_Power_Boxes_1.jpg}
  	\end{center}
	
	\column{0.5\textwidth}
	
	\vspace{-3mm}
	\begin{center}
		\textbf{LED Boxes}
		\includegraphics[width=0.8\textwidth, height=.45\textheight,angle=270]{/home/skbarcus/JLab/SBS/HCal/Documents/HCal_Update_Nov_9_2020/Pictures/LED_Boxes.jpg}
  	\end{center}
	
	\end{columns}

\end{frame}

\begin{frame}{LED System Work Cont.}

    \begin{itemize}
    		\item LEDs control boards use clock and data signals.
        \item New HV cables (thanks Chuck!).
        \item New LED pulse signal cable bundles.
        \item Retimed fADCs and F1TDCs for LEDs.
    \end{itemize}

%    \begin{columns}[T,onlytextwidth]
%	\column{0.5\textwidth}
	
	\vspace{-2mm}
	\begin{center}
	    \textbf{Clock = Blue, Data = Yellow }
	    \includegraphics[width=1\linewidth]{/home/skbarcus/JLab/SBS/HCal/Pictures/LEDs/Good_LED_Clock_and_Data_In.jpg}
  	\end{center}
	
%	\column{0.5\textwidth}
%	
%	\vspace{-2mm}
%	\begin{center}
%		\textbf{Retiming LEDs}
%		\includegraphics[width=1\linewidth]{/home/skbarcus/JLab/SBS/HCal/Pictures/LEDs/LED_Timing_Offset.jpg}
%  	\end{center}
%	
%	\end{columns}

\end{frame}

\begin{frame}{LED Event Display}

	\begin{itemize}
		\item Now able to illuminate all PMTs at once using the LEDs.
		\item Gives a fast stable signal to work with.
	\end{itemize}

	\begin{center}
		\includegraphics[width=1\linewidth]{/home/skbarcus/JLab/SBS/HCal/Pictures/LEDs/LED_Pulses_Clean.png}
  	\end{center}

\end{frame}

\begin{frame}{Voltage Calibration Scans}

	\begin{itemize}
		\item Goal: Set PMT HVs such that each PMT has the same signal response to equal amounts of light (energy deposited in scintillators).
		\item Plot fADC signal strength vs. PMT HV for a stable LED source.
		\item fADC signal strength can be measured by signal peak or integral.
		\begin{itemize}
			\item[--] 250 Hz fADCs (4 ns bins) with 30 samples taken (120 ns window).
		\end{itemize}
		\item Landau function happens to fit pulse well.
	\end{itemize}

	\begin{center}
		\includegraphics[width=0.8\linewidth]{/home/skbarcus/JLab/SBS/HCal/Pictures/Cosmics/Landau_Fit_706_Evt1_2-10_Clean.png}
  	\end{center}

\end{frame}

\begin{frame}{Signal Peak and Integral Proportionality}

	\begin{itemize}
		\item Plot of fADC integral vs fADC signal peak.
			\begin{itemize}
				\item[--] Relationship is linear.
				\item[--] Both fADC signal peak and integral valid but integral is simpler. 
			\end{itemize}
	\end{itemize}
	
	\begin{center}
		\includegraphics[width=0.8\linewidth]{/home/skbarcus/JLab/SBS/HCal/Pictures/LEDs/PMT_2-1_Integral_vs_Peak_Run1222.png}
  	\end{center}

\end{frame}


\begin{frame}{LED Voltage Scan Procedure}

	\begin{itemize}
		\item Program LEDs to cycle during run in 1000 event increments.
			\begin{itemize}
				\item[--] Off$\rightarrow$LED1$\rightarrow$LED2$\rightarrow$LED3$\rightarrow$LED4$\rightarrow$LED5$\rightarrow$Repeat
			\end{itemize}
		\item The LEDs being off provides regular pedestal measurements.
		\item Multiple LEDs are used to ensure PMT HV calibrations are consistent.
		\item Data was taken on the left half of HCal at eight voltages.
			\begin{itemize}
				\item[--] -1200 V, -1250 V, -1300 V, -1350 V, -1375 V, -1400 V, -1425 V, -1450 V.
				\item[--] Five full cycles of data were taken (5k events/setting).
			\end{itemize}
	\end{itemize}
	
	\begin{columns}[T,onlytextwidth]
	\column{0.8\textwidth}
	\begin{itemize}
		\item At low voltages and dimmer LEDs sometimes there is no PMT response.
		\item At high voltage and brighter LEDs sometimes the fADC saturates.
	\end{itemize}
	
	\column{0.2\textwidth}
	\vspace{-4mm}
	\begin{center}
		\includegraphics[width=1.\linewidth]{/home/skbarcus/JLab/SBS/HCal/Pictures/LEDs/Single_Saturated_fADC.png}
  	\end{center}
  	
	\end{columns}

\end{frame}

\begin{frame}{fADC Pedestals}

	\begin{itemize}
		\item Typical fADC pedestal results using raw ADC units (RAU) and summed raw ADC units (SRAU).
			\begin{itemize}
				\item[--] RAU pedestal std. dev. $\approx$1.5 and SRAU std. dev. $\approx$20.
			\end{itemize}
	\end{itemize}

	\begin{columns}[T,onlytextwidth]
	\column{0.5\textwidth}
	\begin{center}
		\includegraphics[width=1.\linewidth]{/home/skbarcus/JLab/SBS/HCal/Pictures/LEDs/fADC_RAU_Pedestal_Clean.png}
  	\end{center}
	
	\column{0.5\textwidth}
	\begin{center}
		\includegraphics[width=1.\linewidth]{/home/skbarcus/JLab/SBS/HCal/Pictures/LEDs/fADC_SRAU_Pedestal_Clean.png}
  	\end{center}
	
	\end{columns}

\end{frame}

\begin{frame}{SRAU for 5 LEDs}

	\begin{itemize}
		\item Fit each LED fADC signal histogram with a Gaussian for each HV setting.
		\item Pictured below PMT 129 at a single voltage looking at 5 different LEDs.
		\begin{itemize}
			\item[--] Each LED roughly doubles in brightness as expected.
			\item[--] LED signals not fit all together. Plot just shows results.
			\item[--] Plot is not pedestal subtracted.
		\end{itemize}
	\end{itemize}

	\begin{center}
		\includegraphics[width=0.9\linewidth]{/home/skbarcus/JLab/SBS/HCal/Pictures/LEDs/5_LED_Peaks_Clean.png}
  	\end{center}

\end{frame}

\begin{frame}{Number of Photoelectrons from LEDs}

\begin{itemize}
	\item Calculate number of photoelectrons produced by LED:
			\begin{equation}
				NPE = \left( \frac{<ADC>}{\sqrt{RMS^2-RMS_{pedestal}^2}} \right)^2.
			\end{equation}
	\begin{itemize}
		\item[--] $<ADC>$ is the mean ADC value (SRAU and pedestal subtracted).
		\item[--] $RMS$ is the standard deviation of the LED spectrum.
		\item[--] $RMS_{pedestal}$ is the standard deviation of the pedestal value.
	\end{itemize}
	\item Looking at PMT 129 again we find the following:
		\begin{itemize}
			\item[--] LED1 = 64 PE.
			\item[--] LED2 = 103 PE.
			\item[--] LED3 = 180 PE.
			\item[--] LED4 = 516 PE.
			\item[--] LED5 = 1356 PE.
		\end{itemize}
	%\item Variation from fiber quality and positioning.
	\item Max energy deposited in single PMT for $G_M^n$ 13.5 GeV kinematic is 700 MeV$\rightarrow$3850 PE.
	\begin{itemize}
		\item[--] Sixth LED produces 3-4k PE so next tests will use that LED.
	\end{itemize}
\end{itemize}	

\end{frame}

\begin{frame}{Fitting the LED Voltage Scan Results}

	\begin{itemize}
		\item The fADC signal strength should go roughly like the high voltage raised to the power of the number of dynodes in the PMTs.
		\item Two different PMT types.
			\begin{itemize}
				\item[--] 192 CMU PMTs = 12 stage 2'' Photonis XP2262 PMTs.
				\item[--] 96 JLab PMTs = 8 stage 2'' Photonis XP2282 PMTs.
			\end{itemize}
	\end{itemize}
	
	\begin{equation}
		SRAU = P_0\left( HV - P_1 \right)^{P_2}-P_3
	\end{equation}
	\begin{itemize}
		\item $HV$ = PMT HV setting.
		\item $P_0$ = coefficient.
		\item $P_1$ and $P_3$ = X and Y offsets.
		\item $P_2$ = exponent corresponding to number of stages.
	\end{itemize}

\end{frame}

\begin{frame}{Selected Results}

	\begin{itemize}
		\item The CMU PMTs have exponents $\approx$ 10-11.
	\end{itemize}
	
	\begin{center}
		\includegraphics[width=1.\linewidth]{/home/skbarcus/JLab/SBS/HCal/Pictures/LEDs/CMU_PMT_Good_Chi2_Clean.png}
  	\end{center}

\end{frame}

%\begin{frame}{Selected Results}
%
%	\begin{itemize}
%		\item Some good looking fits have questionable $\chi^2$. Still investigating these.
%	\end{itemize}
%	
%	\begin{center}
%		\includegraphics[width=1.\linewidth]{/home/skbarcus/JLab/SBS/HCal/Pictures/LEDs/CMU_PMT_Bad_Chi2_Clean.png}
%  	\end{center}
%
%\end{frame}

\begin{frame}{Selected Results}

	\begin{itemize}
		\item The JLab PMTs have exponents $\approx$ 8.
	\end{itemize}
	
	\begin{center}
		\includegraphics[width=1.\linewidth]{/home/skbarcus/JLab/SBS/HCal/Pictures/LEDs/JLab_PMT_Good_Chi2_Clean.png}
  	\end{center}

\end{frame}

\begin{frame}{Saturation}

	\begin{itemize}
		\item Example of the fifth LED causing the fADC to saturate.
	\end{itemize}
	
	\begin{center}
		\includegraphics[width=1.\linewidth]{/home/skbarcus/JLab/SBS/HCal/Pictures/LEDs/CMU_PMT_Saturating_LED_5_Clean.png}
  	\end{center}

\end{frame}

\begin{frame}{Odd Results}

	\begin{itemize}
		\item Then there are the oddball results that need to be individually checked.
	\end{itemize}
	
	\begin{center}
		\includegraphics[width=1.\linewidth]{/home/skbarcus/JLab/SBS/HCal/Pictures/LEDs/CMU_PMT_Odd_Data_Clean.png}
  	\end{center}

\end{frame}

\begin{frame}{Going Forward}

	\begin{itemize}
		\item Coefficient is very small. Possibly making the fit difficult. 
		\item Normalizing HV and signal would make coefficient more similar to other parameters.
		\begin{equation}
			\frac{G_2}{G_1} = \frac{V_2}{V_1}^{P_0 * N_{dynodes}}
		\end{equation}
			\begin{itemize}
				\item[--] $P_0 \approx$ 0.6-0.8. (Photonis PMT Basics.)  
			\end{itemize}
		%\item Try using only one offset parameter.
		%\item Test which model is most effective with model selection criteria.
		\item Optimize fit starting parameters to ensure not trapped in local minima.
		\item Find and remove saturated fADCs. Check other anomalies.
		\item Print out NPE and other parameters to compare PMTs and flag issues.
	\end{itemize}

\end{frame}

\begin{frame}{Cosmic Ray HV Calibration}

	\begin{itemize}
		\item Can also calibrate the HV using cosmic rays.
		\item Want to set cosmic ray signal such that a maximum energy deposition hadron during $G_M^n$ wouldn't saturate the electronics.
		\item There are two considerations for electronics saturation.
			\begin{itemize}
				\item[--] The PS 776 10$\times$ amplifiers which have a 3 V saturation limit $\rightarrow$ no more than 300 mv off the PMT max.
				\item[--] The fADCs have a 2V dynamic range and 4096 (0-4095) channels, or Raw ADC Units (RAU).
				\item[--] So we have 2.0475 RAU/mV.
				\item[--] Max 3 V off amp with about 50\% cable attenuation gives 1.5 V at the fADCs so our amps are our limiting factors.
			\end{itemize}
	\end{itemize}

\end{frame}

\begin{frame}{G4SBS for HV Calibration}

	\begin{itemize}
		\item Need G4SBS to determine how much energy is deposited by hadrons vs cosmics.
		\begin{itemize}
			\item[--] Plot of energy deposited in scintillators vs incident hadron energy. (Plot thanks to Sebastian Seeds.)
		\end{itemize}
	\end{itemize}

	\begin{center}
	\includegraphics[width=0.8\linewidth]{/home/skbarcus/JLab/SBS/HCal/Documents/From_Sebastian/P_and_N_Eng_Dep_11_6_2020/Edep_Einc_HCAL_pn.pdf}
	\end{center}

\end{frame}

\begin{frame}{G4SBS for HV Calibration}

	\begin{itemize}
		\item Plot of maximum energy deposited in scintillators seen by a single PMT for $G_M^n$ max kinematic.
			\begin{itemize}
				\item[--] Max energy deposited in single PMT $\approx$ 700 MeV.
			\end{itemize}
	\end{itemize}

	\begin{center}
	\includegraphics[width=1.0\linewidth]{/home/skbarcus/JLab/SBS/HCal/Documents/Energy_Deposition/TeX_Files/Sumedep_per_Hit_13_5.png}
	\end{center}

\end{frame}

\begin{frame}{G4SBS for HV Calibration}

	\begin{itemize}
		\item Now we need the energy deposited by cosmic rays so that we can set a scale for our HV.
		\begin{itemize}
			\item[--] Plots show average energy deposited in a single PMT's scintillators for a vertical cosmic ray. (Plots from Juan Carlos Cornejo.)
			\item[--] Average energy $\approx$ 14 MeV.
		\end{itemize}
	\end{itemize}

    \begin{columns}[T,onlytextwidth]
	\column{0.5\textwidth}
	
	\vspace{-5mm}
	\begin{center}
	\includegraphics[width=0.5\linewidth]{/home/skbarcus/JLab/SBS/HCal/Documents/Cosmics/Cosmic_Simulations_Cuts_JC_1-3.png}
	\end{center}
	
	\column{0.5\textwidth}
	
	\vspace{-5mm}
	\begin{center}
	\includegraphics[width=0.5\linewidth]{/home/skbarcus/JLab/SBS/HCal/Documents/Cosmics/Cosmic_Simulations_Cuts_JC_4-6.png}
	\end{center}
	
	\end{columns}

\end{frame}

\begin{frame}{Cosmic HV Calibration}

	\begin{itemize}
		\item Maximum energy deposited by a hadron is 50 times greater than that of the average cosmic ray (700 MeV/14 MeV = 50).
		\item 300 mV maximum off the PMTs to not saturate the amps $\rightarrow$ 1/50th of max is 6 mV off of the PMTs for an average cosmic ray.
		\item 6 mV from the PMT is amplified to 60 mV off the amp.
		\item 60 mV then attenuates to $\approx$30 mV at the fADCs.
		\item 30 mV at the fADC is equal to 61 RAU.
		\item If we set all HVs such that the average cosmic ray produces 61 RAU in the fADC then the maximum energy hadrons in $G_M^n$ will not saturate the electronics.
	\end{itemize}

\end{frame}

\begin{frame}{Cosmic HV Calibration}

	\begin{itemize}
		\item Cosmic runs are currently being taken to calibrate the HV based on fADC signal.
		\item Plot shows what G4SBS predicts cosmics will look like in HCal. (Plot from Juan Carlos Cornejo.)
	\end{itemize}
	
	\begin{center}
	\includegraphics[width=1.\linewidth]{/home/skbarcus/JLab/SBS/HCal/Pictures/Cosmics/G4SBS_Cosmics.png}
	\end{center}

\end{frame}

\begin{frame}{Actaul Cosmic Results}

	\begin{itemize}
		\item Plot shows all PMT hits for the whole detector.
		\begin{itemize}
			\item[--] Right half of HCal (0-143) needs some channels checked.
			\item[--] Looks like a bad discriminator is likely the issue.
			\item[--] Otherwise resembles G4SBS well.
		\end{itemize}
		\item Once issues are resolved HV calibration will continue.
	\end{itemize}

	\begin{center}
	\includegraphics[width=1.\linewidth]{/home/skbarcus/JLab/SBS/HCal/Pictures/Cosmics/Vertical_TDC_Hits_Run1239_Clean.png}
	\end{center}

\end{frame}

\begin{frame}{Acknowledgments}
Thanks to \alert{Gregg Franklin} for his many dedicated years designing and overseeing the construction of {\hcal}. Thanks to \alert{Universit\'{a} di Catania} for their major financial contributions. Many other people and institutions were involved in making {\hcal} possible, including, but not limited to:
    \begin{itemize}
        \item Thanks to the many students who have worked on {\hcal} including \setbeamercolor{alerted text}{fg=TolDarkBlue}\alert{Alexis Ortega}, \alert{So Young Jeon}, \alert{Jorge Pe\~{n}a}, \alert{Carly Wever}, and  \alert{Dimitrii Nikolaev}. 
        \item Thanks to our current students \setbeamercolor{alerted text}{fg=TolLightRed}\alert{Vanessa Brio} and \alert{Sebastian Seeds}.
        \item Thanks to \setbeamercolor{alerted text}{fg=mLightBrown}\alert{Alexandre Camsonne} for helping us get the DAQ working and finding all the modules for us.
        \item Thanks to \alert{Chuck Long} for all his help fixing and acquiring things.
        \item Thanks to \alert{Bryan Moffit} for DAQ help.
        \item Thanks also to \alert{Brian Quinn} and \alert{Bogdan Wojtsekhowski}.
        \item Thanks to \setbeamercolor{alerted text}{fg=TolDarkBlue}\alert{Vanessa Brio}, \alert{Cattia Petta}, and \alert{Vincenzo Bellini} for their cosmic commissioning efforts.% last summer.
        %\item Finally welcome to \setbeamercolor{alerted text}{fg=TolLightRed}\alert{Dimitrii Nikolaev}, \alert{Jim Napolitano}, \alert{Donald Jones}, and \alert{Kent Paschke}.
    \end{itemize}

\end{frame}

\end{document}
