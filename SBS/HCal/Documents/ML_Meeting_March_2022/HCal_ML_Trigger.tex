\documentclass[10pt]{beamer}

%Allow captions for subfigures
\usepackage{subcaption}

%Setup for bibtex.
\usepackage[%
  sorting=none,%
%  backend=bibtex,%
  backend=biber,%
%  style=numeric,%
  style=authoryear,% Author , Date style citations.
  %style=phys,% APS Style citations.
  pageranges=false,% Print only first page, only works with style=phys
  chaptertitle=false,% for incollections show chapter titles - true for AIP style, false for APS style
  articletitle=true,% Print article title, AIP = false, APS = true
  biblabel=brackets,% Number entries by bracked notation
  related=true,%
  isbn=false,% Don't print ISBN
  doi=false,% Don't print DOI
  url=false,% Don't print URL
  eprint=false,% Don't print eprint information
  hyperref=true,%
%  note=false,%
  firstinits=true,% Use first intials only
  maxnames=3,% Truncate after N names
  minnames=1,% Print at least N names if truncated
%  natbib=true%
 ]{biblatex}
 
%Italicize et al.
 \DefineBibliographyStrings{english}{%
  andothers = {\textit{et al}\adddot}
}

%\addbibresource{bib_Books.bib}
\addbibresource{references.bib}

%\begin{filecontents}{\jobname.bib}
%@article{Baird2002,
%author = {Baird, Kevin M and Hoffmann, Errol R and Drury, Colin G},
%journal = {Applied ergonomics},
%month = jan,
%number = {1},
%pages = {9--14},
%title = {{The effects of probe length on Fitts' law.}},
%volume = {33},
%year = {2002}
%}
%\end{filecontents}

\usetheme{metropolis}
\usepackage{appendixnumberbeamer}

\usepackage{booktabs}
\usepackage[scale=2]{ccicons}

\usepackage{pgfplots}
\usepgfplotslibrary{dateplot}

\usepackage{xspace}
\newcommand{\themename}{\textbf{\textsc{metropolis}}\xspace}

%Allows for the making of cells in tables.
\usepackage{makecell}

\usepackage{amsmath}

%\setbeamertemplate{itemize item}[circle]
%\setbeamertemplate{itemize subitem}[-]

\newcommand{\brown}{\setbeamercolor{alerted text}{fg=mLightBrown}}
\newcommand{\red}{\setbeamercolor{alerted text}{fg=TolLightRed}}
\newcommand{\blue}{\setbeamercolor{alerted text}{fg=TolDarkBlue}}

\title{HCal Neural Network Real-Time Trigger}
\subtitle{}
\date{March 17\textsuperscript{th} 2022}
\author{Scott Barcus}
%\institute{Jefferson Lab}
% \titlegraphic{\hfill\includegraphics[height=1.5cm]{logo.pdf}}

\begin{document}

\maketitle

\begin{frame}{Commissioning a Machine Learning Detector Trigger for HCal}

	\begin{itemize}
		\item \alert{Motivation:}
			\begin{itemize}
				\item[--] High background rates obscure physics signals.
			\end{itemize}
		\item \alert{Traditional Solutions:}
			\begin{itemize}
				\item[--] Prescaling the data.
				\item[--] Energy threshold cuts.
				\item[--] Decreasing the beam current.
			\end{itemize}
		\item \alert{Machine Learning Solution:}
			\begin{itemize}
				\item[--] Use data from Geant4 converted to detector outputs to train NN.
				\item[--] Train a neural network to classify detector events (e.g. p, n, $\pi$).
				\item[--] Load trained NN onto VTP FPGA (fast) to use as HCal trigger.
			\end{itemize}
		\item \alert{Goal:}
			\begin{itemize}
				\item[--] Show that NNs on FPGAs can be used as triggers for JLab detectors.
				\item[--] Compare HCal ML trigger to traditional triggering methods.
			\end{itemize}
		\item \alert{Tools:}
			\begin{itemize}
				\item[--] Python, Numpy, Scikit-learn, Tensorflow, Keras, Google Colaboratory (GPUs), hls4ml, ROOT.
			\end{itemize}
	\end{itemize}

\end{frame}

\begin{frame}{VXS Trigger Processor}

    \begin{columns}[T,onlytextwidth]
  	\column{0.55\textwidth}
	\begin{itemize}
		\item Linux OS which can run standard CODA ROC (Zynq-7030 SoC).
		\item \alert{Create software trigger with FPGA (like ROC).}
			\begin{itemize}
				\item[--] Virtex-7 7VX550T FPGA.
			\end{itemize}
	\end{itemize}

	\column{0.4\textwidth}	
	
  	\centerline{\includegraphics[width=1.\textwidth]{/home/skbarcus/JLab/SBS/HCal/Documents/ML_Meeting_March_2022/pictures/VTP_FPGA_Stats.png}}
  	
  	\end{columns}
  	
  	\centerline{\includegraphics[width=0.75\textwidth]{/home/skbarcus/JLab/SBS/HCal/Documents/ML_Meeting_March_2022/pictures/VTP_Scematic.png}}

\end{frame}

\begin{frame}{Data Acquisition System}

    \begin{columns}[T,onlytextwidth]
  	\column{0.55\textwidth}
  	
  	\begin{itemize}
  		\item Two VXS crates.
  		\item \setbeamercolor{alerted text}{fg=TolLightRed}\alert{18 16-channel fADC250 flash ADCs measure energy.}
  			\begin{itemize}
  				\item[--] Takes numerous samples (250 MHz, 4ns).
  				\item[--] Time over threshold measurements extract timing.% (CFD removes time walk).
  			\end{itemize}
  		\item \setbeamercolor{alerted text}{fg=TolDarkBlue}\alert{5 64-channel F1TDCs for timing.}
  		\item VXS Trigger Processors (VTPs) contain FPGAs to form triggers (future use).
  		\item \setbeamercolor{alerted text}{fg=mLightBrown}\alert{Triggers:}
  			\begin{itemize}
  				\item[--] Scintillator paddle (cosmics).
  				\item[--] Summing module trigger.
  				\item[--] LED pulser trigger.
  				\item[--] BigBite coincidence trigger.
  			\end{itemize}
  	\end{itemize}
  	
  	\column{0.5\textwidth}
  	\vspace{-2mm}
	\centerline{\includegraphics[width=0.65\textwidth]{/home/skbarcus/JLab/SBS/HCal/Pictures/Cosmics/Landau_Fit_706_Evt1_2-10_Clean.png}}
	
	\vspace{1mm}	
	
	\centerline{\includegraphics[width=0.85\textwidth]{/home/skbarcus/JLab/SBS/HCal/Pictures/Cosmics/Cosmic_Hit_run820_evt16_Arrow.png}}
	
	\vspace{1mm}
	
	\centerline{\includegraphics[width=0.4\textwidth]{/home/skbarcus/Documents/JLab_SS1/Seminar/Summing_Module_Triggers.png}}

	\end{columns}

\end{frame}

\begin{frame}{Neural Network Preprocessing}

	\begin{itemize}
		%\item Use fADC waveform integral initially.
		\item $\approx$4M events from $G_M^n$ experiment used for training.
		\begin{itemize}
			\item[--] Experimental trigger was electron arm.% so these are basically coincident trigger events.
			\item[--] \alert{Training data are HCal cluster (X,Y) position, energy (fADC integrals), and TDC time (electron arm info. improves).} 
			\item[--] Events include elastic hadrons (signal) as well as inelastic events, knock on electrons, pions, cosmics etc. (background). 
		\end{itemize}
		\item Normalize the training data.
		\item Identify elastic events with traditional physics cuts on calorimeter energies, position, $W^2$ etc. \setbeamercolor{alerted text}{fg=TolLightRed}\alert{$\approx$ 8k elastic hadrons.}
			\begin{itemize}
				\item[--] Small signal makes training on whole data set difficult. Remove 95-99\% non-elastic events to help train.
			\end{itemize}
		\item Split data into training, validation, and test sets.
		\begin{itemize}
			\item[--] \setbeamercolor{alerted text}{fg=mLightBrown}\alert{Training: Labeled data that trains the CNN (80\%).}
			\item[--] \setbeamercolor{alerted text}{fg=TolDarkBlue}\alert{Validation: Data to prevent overfitting (10\% of training).}
			\item[--] \setbeamercolor{alerted text}{fg=TolLightRed}\alert{Test: Partitioned data to test NN's ability to extrapolate (20\%).}
		\end{itemize}
	\end{itemize}

\end{frame}

\begin{frame}{Build Neural Network Architecture}

\begin{columns}[T,onlytextwidth]
  	\column{.55\textwidth}
  	
  	\begin{itemize}
  		\item 4 variables cluster (X,Y), energy, and timing.
  		\item \alert{Four layer fully connected neural network:}
  		\begin{itemize}
  			\item[--] 512 neurons, activation = relu.
  			\item[--] 1024 neurons, activation = relu.
  			\item[--] 512 neurons, activation = relu.
  			\item[--] 1 neuron, activation = sigmoid (0-1 output).
  		\end{itemize}
  	\end{itemize}

	\column{.45\textwidth}
    \begin{center}
    		\includegraphics[width=1.\linewidth]{/home/skbarcus/JLab/SBS/HCal/Machine_Learning/GMN/Pictures/GMn_SBS4_Elastic_NN_Architecture_1_Clean.png}
    	\end{center}
    		
    	\end{columns}	
    		
    	\begin{center}
        \includegraphics[width=1.\linewidth]{/home/skbarcus/JLab/SBS/HCal/Machine_Learning/GMN/Pictures/GMn_SBS4_Elastic_NN_Architecture_2_Clean.png}
	\end{center}
	
\end{frame}

\begin{frame}{Train Neural Network}

	\vspace{-1mm}
	\begin{itemize}
		\item \alert{Neural networks are  trained by minimizing a loss function.}
			\begin{itemize}
				\item[--] Use mean squared error loss function.
				\item[--] Optimizer Adam with lr = 0.0001. Batch size = 64. 150 epochs.
			\end{itemize}
		%\item \setbeamercolor{alerted text}{fg=TolDarkBlue}\alert{Validation data} = 10\% of \setbeamercolor{alerted text}{fg=TolLightRed}\alert{training data}.
		\item Loss function NN minimizes is decreasing with training.
		\begin{itemize}
			\item[--] 10\% validation data (monitor overfitting).
			\item[--] \setbeamercolor{alerted text}{fg=TolLightRed}\alert{Validation loss has yet to diverge from training data loss. May be able to train longer without overfitting.}
		\end{itemize}
	\end{itemize}
	
	\vspace{-3mm}
	\begin{center}
    		\includegraphics[width=1.\linewidth]{/home/skbarcus/JLab/SBS/HCal/Machine_Learning/GMN/Pictures/GMn_SBS4_Elastic_NN_Training_Plots.png}
    	\end{center}
	
\end{frame}

\begin{frame}{Evaluate Neural Network Performance Cont.}
	\vspace{-2mm}
	\begin{itemize}
		\item Run test data through NN and evaluate predictions.
		\item Confusion matrix:
	\end{itemize}
	
	\begin{center}
    		\includegraphics[width=0.4\linewidth]{/home/skbarcus/JLab/SBS/HCal/Machine_Learning/GMN/Pictures/GMn_SBS4_Elastic_NN_Confusion_Matrix.png}
    	\end{center}
    	
    	\vspace{-5mm}
	\begin{itemize}
		\item \alert{Predicts 82.59\% of events correctly, but elastic events matter more.}
		%\item \alert{NN elastic trigger would decrease data rate by a factor of 2.98.}
		\item Correctly identified  82\% of non-elastics as non-elastics.
		\item Incorrectly identified 18\% of non-elastics as elastics.
		\begin{itemize}
			\item[--] Not a big deal if overall data rate still reduced.
		\end{itemize}
		\item \setbeamercolor{alerted text}{fg=TolLightRed}\alert{Incorrectly identified  15.46\% of elastics as non-elastics.}
		\begin{itemize}
			\item[--] Dangerous if losing biased elastic sample. Less so if random.
		\end{itemize}
		\item \setbeamercolor{alerted text}{fg=TolDarkBlue}\alert{Correctly identified  84.54\% of elastics as elastics.}
	\end{itemize}
	
\end{frame}

\begin{frame}{Any Patterns in NN Mistakes?}

	\begin{itemize}
		\item Are there any patterns in the events the NN is misclassifying?
		\item Plot the labeled events on HCal's surface (left) then compare with the NN classifications (right).
			\begin{itemize}
				\item[--] Left: \setbeamercolor{alerted text}{fg=TolLightRed}\alert{red = non-elastic}, \setbeamercolor{alerted text}{fg=TolDarkBlue}\alert{blue = elastic}.
				\item[--] Right: \setbeamercolor{alerted text}{fg=TolDarkBlue}\alert{blue = correct elastic}, \setbeamercolor{alerted text}{fg=TolLightRed}\alert{red = incorrect elastic}, \setbeamercolor{alerted text}{fg=mLightGreen}\alert{green = correct nonelastic}, \setbeamercolor{alerted text}{fg=mLightBrown}\alert{yellow = incorrect nonelastic}.
			\end{itemize}
	\end{itemize}
	
	\vspace{-7mm}
	\begin{columns}[T,onlytextwidth]
  	\column{.5\textwidth}
	
	\begin{center}
    		\includegraphics[width=0.6\linewidth]{/home/skbarcus/JLab/SBS/HCal/Machine_Learning/GMN/Pictures/GMn_SBS4_Elastic_NN_Plot_Truth.png}
    	\end{center}
    	
    	\column{.5\textwidth}
    	
    	\begin{center}
    		\includegraphics[width=0.6\linewidth]{/home/skbarcus/JLab/SBS/HCal/Machine_Learning/GMN/Pictures/GMn_SBS4_Elastic_NN_Plot_Predicted.png}
    	\end{center}
    	
    	\end{columns}

\end{frame}

\begin{frame}{Individual Training Variable Comparisons}
	\begin{center}
	
%	\onslide<1>\item Cluster energy of NN classified events.
%	\onslide<2>\item Cluster timing of NN classified events.
%	\onslide<3>\item Cluster X-position of NN classified events.
%	\onslide<4>\item Cluster Y-position of NN classified events.

	\begin{overprint}%[11.7cm]
	\onslide<1>\centerline{\includegraphics[width=1.\linewidth]	{/home/skbarcus/JLab/SBS/HCal/Machine_Learning/GMN/Pictures/GMn_SBS4_Elastic_NN_Cluster_Energy.png}}
	\centerline{\bf{Cluster energy of NN classified events}}
	\onslide<2>\centerline{\includegraphics[width=1.\linewidth]	{/home/skbarcus/JLab/SBS/HCal/Machine_Learning/GMN/Pictures/GMn_SBS4_Elastic_NN_Time.png}}
	\centerline{\bf{Cluster timing of NN classified events.}}
	\onslide<3>\centerline{\includegraphics[width=1.\linewidth]	{/home/skbarcus/JLab/SBS/HCal/Machine_Learning/GMN/Pictures/GMn_SBS4_Elastic_NN_X-Position.png}}
	\centerline{\bf{Cluster X-position of NN classified events.}}
	\onslide<4>\centerline{\includegraphics[width=1.\linewidth]	{/home/skbarcus/JLab/SBS/HCal/Machine_Learning/GMN/Pictures/GMn_SBS4_Elastic_NN_Y-Position.png}}
	\centerline{\bf{Cluster Y-position of NN classified events.}}
	\end{overprint}
	\end{center}
	
\end{frame}

\begin{frame}{Future Work with HCal Neural Networks}

	\begin{itemize}
		\item \setbeamercolor{alerted text}{fg=mLightBrown}\alert{Train on more beam data!}
			\begin{itemize}
				\item[--] Cluster finding AI like K-means and DBSCAN \parencite{Article:osti_1786294}.
				\item[--] Can protons and neutrons be separated?
				\item[--] Evaluate trigger biases.
			\end{itemize}
		\item \setbeamercolor{alerted text}{fg=TolDarkBlue}\alert{Full minimum bias Geant4 simulations for upcoming experiments}.
		\begin{itemize}
			\item[--] Utilize full fADC waveform (\setbeamercolor{alerted text}{fg=TolLightRed}\alert{resource intensive}).
		\end{itemize}
		\item \setbeamercolor{alerted text}{fg=mLightBrown}\alert{Incorporate other detector data.}
		\begin{itemize}
			\item[--] Use timing information form VETROC TDCs.
			\item[--] Explore using high-level info from BigBite electron detector. %(e$^-$ arm).
		\end{itemize}
		\item \setbeamercolor{alerted text}{fg=TolDarkBlue}\alert{Explore other NN architectures and convolutional NNs.}
		\item \setbeamercolor{alerted text}{fg=mLightBrown}\alert{Load NN onto VTP FPGA and use as real-time HCal trigger.}
		\begin{itemize}
			\item[--] Use hls4ml to translate the NN into FPGA compatible code.
			\item[--] Test with beam in JLab's Hall A.
		\end{itemize}
	\end{itemize}

\end{frame}

\begin{frame}{Questions for the Data Science Department}

	\begin{itemize}
		\item How to evaluate the size and speed of a NN that can be placed on an FPGA?
		\item How difficult is translating a NN to FPGA compatible code?
			\begin{itemize}
				\item[--] Is hls4ml the best approach here?
			\end{itemize}
		\item What is the best way to evaluate the performance of the NNs?
		\item What is the best way to explore different model architectures and hyperparameters thoroughly?
		\item What other ML/AI techniques could be of benefit to an online trigger?
		\item How can one access JLab's GPU resources for training etc.?
		\item Other ML/AI detector applications like auto-calibrations?
	\end{itemize}

\end{frame}

\end{document}