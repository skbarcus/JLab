\documentclass[10pt]{beamer}

%Allow captions for subfigures
\usepackage{subcaption}

%Setup for bibtex.
\usepackage[%
  sorting=none,%
%  backend=bibtex,%
  backend=biber,%
%  style=numeric,%
  style=phys,% 
  pageranges=false,% Print only first page, only works with style=phys
  chaptertitle=false,% for incollections show chapter titles - true for AIP style, false for APS style
  articletitle=true,% Print article title, AIP = false, APS = true
  biblabel=brackets,% Number entries by bracked notation
  related=true,%
  isbn=false,% Don't print ISBN
  doi=false,% Don't print DOI
  url=false,% Don't print URL
  eprint=false,% Don't print eprint information
  hyperref=true,%
%  note=false,%
  firstinits=true,% Use first intials only
  maxnames=3,% Truncate after N names
  minnames=1,% Print at least N names if truncated
%  natbib=true%
 ]{biblatex}
 
%Italicize et al.
 \DefineBibliographyStrings{english}{%
  andothers = {\textit{et al}\adddot}
}

%\addbibresource{bib_Books.bib}

%\begin{filecontents}{\jobname.bib}
%@article{Baird2002,
%author = {Baird, Kevin M and Hoffmann, Errol R and Drury, Colin G},
%journal = {Applied ergonomics},
%month = jan,
%number = {1},
%pages = {9--14},
%title = {{The effects of probe length on Fitts' law.}},
%volume = {33},
%year = {2002}
%}
%\end{filecontents}

\usetheme{metropolis}
\usepackage{appendixnumberbeamer}

\usepackage{booktabs}
\usepackage[scale=2]{ccicons}

\usepackage{pgfplots}
\usepgfplotslibrary{dateplot}

\usepackage{xspace}
\newcommand{\themename}{\textbf{\textsc{metropolis}}\xspace}

%Allows for the making of cells in tables.
\usepackage{makecell}

\usepackage{amsmath}
%\setbeamertemplate{itemize item}[circle]
%\setbeamertemplate{itemize subitem}[-]

\title{HCal Update}
\subtitle{}
\date{November 14\textsuperscript{th} 2019}
\author{Scott Barcus and Juan Carlos Cornejo}
\institute{Jefferson Lab}
% \titlegraphic{\hfill\includegraphics[height=1.5cm]{logo.pdf}}

\begin{document}

\maketitle

\begin{frame}{HCal Hall Layout}

	\begin{itemize}
		\item Detect multiple GeV protons and neutrons. 
	\end{itemize}

%	\begin{figure}[!ht]
	\begin{center}
	\includegraphics[width=1.0\linewidth]{/home/skbarcus/JLab/SBS/HCal/Schematics/HCal_Hall_Layout_Clean.png}
	\end{center}
%	\caption{
%	{\bf{HCal Layout in Hall A.}} }
%	\label{fig:hall_layout}
%	\end{figure}
\end{frame}

\begin{frame}{Detector HV and PMT Cables}

	\begin{itemize}
		\item Detector cabling completed.
	\end{itemize}
	
	\begin{center}
  		\includegraphics[width=0.9\linewidth]{/home/skbarcus/JLab/SBS/HCal/Pictures/20190508_170610.jpg}
  	\end{center}

\end{frame}

\begin{frame}{HCal Front End}

	\includegraphics[width=1.0\textwidth]{/home/skbarcus/JLab/SBS/HCal/Documents/SBSAugust2019_HCAL_StatusReport/HCal_FE.png}

\end{frame}

\begin{frame}{HCal Front End Cont.}

    \begin{columns}[T,onlytextwidth]
  	\column{0.5\textwidth}
	\includegraphics[width=.85\textwidth]{/home/skbarcus/JLab/SBS/HCal/Pictures/HCal_FE_RR1.jpg}
	
	\column{0.5\textwidth}
	\includegraphics[width=.85\textwidth]{/home/skbarcus/JLab/SBS/HCal/Pictures/HCal_FE_RR3.jpg}
	\end{columns}

\end{frame}

\begin{frame}{HCal Front End Cont.}
    \centering
	\includegraphics[width=0.43\textwidth]{/home/skbarcus/JLab/SBS/HCal/Pictures/HCal_FE_RR2.jpg}

\end{frame}

\begin{frame}{HCal DAQ Side}

	\includegraphics[width=1.0\textwidth]{/home/skbarcus/JLab/SBS/HCal/Documents/SBSAugust2019_HCAL_StatusReport/HCal_DAQ.png}

\end{frame}

\begin{frame}{HCal DAQ Side Cont.}

    \begin{columns}[T,onlytextwidth]
  	\column{0.5\textwidth}
	\includegraphics[width=.85\textwidth]{/home/skbarcus/JLab/SBS/HCal/Pictures/HCal_DAQ_RR4.jpg}
	
	\column{0.5\textwidth}
	\includegraphics[width=.85\textwidth]{/home/skbarcus/JLab/SBS/HCal/Pictures/HCal_DAQ_RR5.jpg}
	\end{columns}

\end{frame}

%\begin{frame}{HCal Front-End Layout}
%
%	\begin{itemize}
%		\item Front end layout.
%	\end{itemize}
%
%	\vspace{-3mm}
%%	\begin{figure}[!ht]
%	\begin{center}
%	\includegraphics[width=0.9\linewidth]{/home/skbarcus/JLab/SBS/HCal/Documents/HCal_Update/HCal_Front_End_Clean.png}
%	\end{center}
%%	\caption{
%%	{\bf{HCal Front-End Layout.}} }
%%	\label{fig:front_end}
%%	\end{figure}
%\end{frame}

%\begin{frame}{HCal Front-End Layout RR1 $\&$ RR3}
%	
%	\begin{columns}[T,onlytextwidth]
%	\column{0.5\textwidth}
%	\vspace{6mm}
%	\begin{itemize}
%		\item RR1 and RR3 are nearly identical and contain:
%			\begin{itemize}
%				\item[--] Bottom crate \alert{9 amplifiers}. 16 ch in 16X2 ch out. Outputs to FADC and splitter panels.
%				\item[--] \alert{9 splitter panels}. 16 ch in 16X2 ch out.  A = output to summers, B = input from amps, C = output to discriminators. 
%				\item[--] \setbeamercolor{alerted text}{fg=TolLightRed}\alert{RR1} top crate has \alert{8 summers + 1 discriminator}. \setbeamercolor{alerted text}{fg=TolDarkBlue}\alert{RR3} top crate has \alert{7 summers}.\setbeamercolor{alerted text}{fg=mLightBrown}
%			\end{itemize}
%	\end{itemize}
%	
%  	\column{0.5\textwidth}
%  	
%  		\begin{center}
%  		\includegraphics<1>[width=0.7\linewidth]{/home/skbarcus/JLab/SBS/HCal/Pictures/20190508_170713.jpg}
%  		\includegraphics<2>[width=0.7\linewidth]{/home/skbarcus/JLab/SBS/HCal/Pictures/20190508_170802.jpg}
%  		\end{center}
%  		
%  		%\begin{center}
%		%\begin{figure}[!ht]
%		%\begin{overprint}[12cm]
%		
%%		\onslide<1>\includegraphics[width=0.3\linewidth]	{/home/skbarcus/JLab/SBS/HCal/Pictures/20190508_170713.jpg}
%%		\caption{{\bf{RR1 Uncabled.}} }
%%		\label{fig:RR1}
%%		\onslide<2>\includegraphics[width=0.3\linewidth]	{/home/skbarcus/JLab/SBS/HCal/Pictures/20190508_170802.jpg}
%%		\caption{{\bf{RR2 Partially Cabled}} }
%%		\label{fig:RR2}
%		
%%		\includegraphics<1>[width=0.3\linewidth]{/home/skbarcus/JLab/SBS/HCal/Pictures/20190508_170713.jpg} \caption{{\bf{RR1 Uncabled.}} }
%%		\label{fig:RR1}
%%		\includegraphics<2>[width=0.3\linewidth]	{/home/skbarcus/JLab/SBS/HCal/Pictures/20190508_170802.jpg}%\captionof{figure}{RR2} 
%%		%\caption{{\bf{RR2 Partially Cabled}} }
%%		\label{fig:RR2}
%		
%		%\end{overprint}
%%		\end{figure}
%		%\end{center}
%  	
%%	\begin{figure}[!ht]
%%	\begin{center}
%%	\includegraphics[width=0.7\linewidth]{/home/skbarcus/JLab/SBS/HCal/Pictures/20190508_170713.jpg}
%%	\end{center}
%%	\caption{
%%	{\bf{RR1 Uncabled.}} }
%%	\label{fig:front_end}
%%	\end{figure}
%	
%	\end{columns}
%	
%\end{frame}

%\begin{frame}{HCal Front-End Layout RR2}
%
%	\begin{columns}[T,onlytextwidth]
%	\column{0.5\textwidth}
%	\vspace{8mm}
%	\begin{itemize}
%		\item RR2 takes inputs from RR1 and RR3.
%		\item RR2 contains:
%		\begin{itemize}
%			\item[--] An upper and lower crate each with 9 discriminators.
%			\begin{enumerate}
%			
%				\item \setbeamercolor{alerted text}{fg=TolLightRed}\alert{RR1} uses the \alert{bottom 9 discriminators}. 
%				\item \setbeamercolor{alerted text}{fg=TolDarkBlue}\alert{RR3} uses the \alert{top 9 discriminators}. 
%			
%			\end{enumerate}
%			\item[--] 10 16X4 ch patch panels (still need one more).
%			\begin{enumerate}
%				\item Bottom 5 PPs go to \alert{FADC}.
%				\item Top 5 PPs go to \alert{TDC}.
%			\end{enumerate}
%		\end{itemize}
%	\end{itemize}
%	
%	\column{0.5\textwidth}
%	  	\begin{center}
%  		\includegraphics[width=0.7\linewidth]{/home/skbarcus/JLab/SBS/HCal/Pictures/20190508_170743.jpg}
%  		\end{center}
%	\end{columns}
%
%\end{frame}

\begin{frame}{Current Status}

	\begin{itemize}
		\item \alert{Detector}:
			\begin{itemize}
				\item[--] Cabling completed.
				\item[--] PMTs need gain matching (some measurements already made).
			\end{itemize}
		\item \alert{Front-End}:
			\begin{itemize}
				\item[--] Electronics are tested (some minor repairs outstanding) except summing modules.
				\item[--] Half of FE is cabled (several cables being repaired still).
			\end{itemize}
		\item \alert{DAQ Side}:
			\begin{itemize}
				\item[--] 3/4 cabled.
				\item[--] Electronics tested and operational.
				\item[--] 18 operational fADCs.
				\item[--] 5 operational F1TDCs.
			\end{itemize}
		\item \alert{Software}:
			\begin{itemize}
				\item[--] Decoder operational.
				\item[--] Still on CODA 2.
				\item[--] Preliminary event-by-event display.
				\item[--] Timing resolution analysis instituted. 
			\end{itemize}
		\item \alert{Cosmic tests} are under way for calibrating the detector.
	\end{itemize}

\end{frame}

\begin{frame}{Event-by-Event Display}

	\begin{itemize}
		\item Selected cosmic events. (Module 4-1 is a reference channel.)
	\end{itemize}

	\begin{center}
  		\includegraphics<1>[width=1.\linewidth]{/home/skbarcus/JLab/SBS/HCal/Pictures/Display_run527_evt375.png}
  		\includegraphics<2>[width=1.\linewidth]{/home/skbarcus/JLab/SBS/HCal/Pictures/Display_run527_evt387.png}
  		\includegraphics<3>[width=1.\linewidth]{/home/skbarcus/JLab/SBS/HCal/Pictures/Display_run527_evt383.png}
  	\end{center}

\end{frame}

\begin{frame}{Timing Resolution Cuts}

	\begin{columns}[T,onlytextwidth]
	\column{0.4\textwidth}
	\begin{itemize}
		\item Require cosmic to be nearly \alert{vertical}.
			\begin{itemize}
				\item[--] Three vertical TDC and fADC signals.
				\item[--] No signals in surrounding modules.
			\end{itemize}
		\item TDC time:
		\begin{equation*}
			\text{T}_{\text{cor}}=\text{T}_{\text{meas}} - \frac{\text{TDC\;1}+\text{TDC\;2}}{2}
		\end{equation*}
		\item Create histogram of all times and find standard deviation.
	\end{itemize}
	
	\column{0.6\textwidth}
	\begin{center}
  		\includegraphics[width=1.3\linewidth]{/home/skbarcus/JLab/SBS/HCal/Pictures/Timing_Resolution_Cuts.png}
  	\end{center}
  	\end{columns}

\end{frame}

\begin{frame}{fADC Integral Plots}

	\begin{itemize}
		\item Events at zero are pedestals. fADC recorded when TDC fired.
	\end{itemize}

	\begin{center}
  		\includegraphics<1>[width=1.\linewidth]{/home/skbarcus/JLab/SBS/HCal/Pictures/fADC_Integrals_540_Row9.png}
  		\includegraphics<2>[width=1.\linewidth]{/home/skbarcus/JLab/SBS/HCal/Pictures/fADC_Integrals_540_Row3.png}
  		\includegraphics<3>[width=1.\linewidth]{/home/skbarcus/JLab/SBS/HCal/Pictures/fADC_Integrals_540_Row11.png}
  	\end{center}

\end{frame}

\begin{frame}{Timing Resolution Plots}

	\begin{itemize}
		\item Average timing resolution is 1-1.5 ns for each module.
	\end{itemize}

	\begin{center}
  		\includegraphics<1>[width=1.\linewidth]{/home/skbarcus/JLab/SBS/HCal/Pictures/Timing_Resolution_540_Row9.png}
  		\includegraphics<2>[width=1.\linewidth]{/home/skbarcus/JLab/SBS/HCal/Pictures/Timing_Resolution_540_Row3.png}
  		\includegraphics<3>[width=1.\linewidth]{/home/skbarcus/JLab/SBS/HCal/Pictures/Timing_Resolution_540_Row11.png}
  	\end{center}

\end{frame}

%\begin{frame}{Cable Tie Bars}
%
%	\begin{itemize}
%		\item Bars help \alert{support cables} without stressing them while also helping with \alert{organization}.
%	\end{itemize}
%	
%	\begin{center}
%  		\includegraphics<1>[width=1.\linewidth]{/home/skbarcus/JLab/SBS/HCal/Pictures/20190506_141426.jpg}
%  		\includegraphics<2>[width=1.\linewidth]{/home/skbarcus/JLab/SBS/HCal/Pictures/20190508_170921.jpg}
%  		\includegraphics<3>[width=1.\linewidth]{/home/skbarcus/JLab/SBS/HCal/Pictures/20190508_170849.jpg}
%    \end{center}
%
%\end{frame}

\begin{frame}{Machine Learning Detector Trigger for HCal LDRD Proposal}

	\begin{itemize}
		\item Motivation:
			\begin{itemize}
				\item[--] High background rates obscure physics signals.
			\end{itemize}
		\item Traditional Solutions:
			\begin{itemize}
				\item[--] Energy threshold cuts.
				\item[--] Prescaling the data.
				\item[--] Decreasing the beam current.
			\end{itemize}
		\item Machine Learning Solution:
			\begin{itemize}
				\item[--] Train a neural network to classify detector events (e.g. p, n, $\pi$).
				\item[--] Use data from G4SBS converted to detector output to train NN.
				\item[--] Load trained NN onto VTP FPGA (fast) to use as HCal trigger.
			\end{itemize}
		\item Goal:
			\begin{itemize}
				\item[--] Demonstrate that NNs can be loaded onto VTPs for triggering JLab detectors.
				\item[--] Allow HCal to run at higher current with a cleaner trigger.
			\end{itemize}
	\end{itemize}

\end{frame}

\begin{frame}{Brief Neural Network Overview}

	\vspace{-3mm}
	\begin{center}
  		\includegraphics<1>[width=0.5\linewidth]{/home/skbarcus/JLab/SBS/HCal/Documents/SBS_Meeting_July_2020/neural_network.png}\\
  		\tiny{Image from https://www.astroml.org/book$\_$figures/chapter9/fig$\_$neural$\_$network.html.}
  	\end{center}
  	
  	\vspace{-5mm}
  	\begin{columns}[T,onlytextwidth]
  	\column{0.5\textwidth}
  	
  	\begin{itemize}
  		\item Feed Forward:
  			\begin{itemize}
  				\item[--] Initialize random weights and biases.
  				\item[--] Enter labeled training data to input layer.
  				\item[--] Calculate activation of each neuron (did it fire?).
  				\item[--] Feed forward activation until output layer reached.
  			\end{itemize}
  	\end{itemize}
  	
  	\column{0.5\textwidth}
  	
  	\begin{itemize}
  		\item Backpropagation:
  			\begin{itemize}
  				\item[--] Evaluate loss function. How wrong is the NN's guess?
  				\item[--] Apply backpropagation algorithm to step back through NN (Chain rule).
  				\item[--] Adjust weights and biases along the gradient of descent (minimize loss function).
  				\item[--] Repeat. 
  			\end{itemize}
  	\end{itemize}

	\end{columns}

\end{frame}

\begin{frame}{Convolutional Neural Networks}

	\begin{center}
  		\includegraphics<1>[width=0.5\linewidth]{/home/skbarcus/JLab/SBS/HCal/Documents/SBS_Meeting_July_2020/CNN.png}\\
  		\tiny{Image from https://www.mdpi.com/2076-3417/9/21/4500.}
  	\end{center}

	\begin{itemize}
		\item Started with image classification. 
		\item Detector traces are essentially images.
			\begin{itemize}
				\item[--] For every event each PMT has numerous fADC samples (like pixels) and a TDC value.
				\item[--] The location of the PMTs relative to one another is important!
			\end{itemize}
		\item Dense NNs assume each neuron connection is equally important.
		\item CNNs scan across the image with a kernel creating filters which identify localized features.
		\item Pooling layers decrease the dimensionality to keep things manageable.
	\end{itemize}

\end{frame}

\begin{frame}{Overview of Research Plan}

	\begin{enumerate}
		\item Produce labeled training data from G4SBS.
		\item Convert G4SBS data into detector signals.
		\item Design model architecture (probably CNN).
		\item Train CNN on simulated detector data.
		\item Optimize model hyperparameters (layers, neurons, learning rate ...).
		\item Using hls4ml translate the CNN into FPGA compatible code.
		\item Load CNN onto VTP FPGA where it will use fADC250 and VETROC (3 requested from LDRD) data to form HCal trigger.
		\item Test the trigger under beam conditions and compare its performance with traditional methods.
	\end{enumerate}

\end{frame}

\begin{frame}{Toy Example HCal Neural Network}

	\begin{itemize}
		\item Create NN to identify events with probable cosmic signals. (Note: traditional methods work better in this illustative example.)
		\item Tools: ROOT, Python, Numpy, Scikit-learn, Tensorflow, Keras, Google Colaboratory (GPUs).
		\item Steps:
			\begin{enumerate}
				\item Format HCal data for NN compatibility.
				\item Split data into train, test, and validate.
			\end{enumerate}
	\end{itemize}

	\begin{center}
  		\includegraphics<1>[width=0.8\linewidth]{/home/skbarcus/JLab/SBS/HCal/Documents/SBS_Meeting_July_2020/Read_Data_Clean.png}
  	\end{center}

\end{frame}

\begin{frame}{Toy Example HCal Neural Network Continued}

  	\begin{columns}[T,onlytextwidth]
  	\column{0.5\textwidth}
  	\begin{center}
  		\includegraphics<1>[width=1.\linewidth]{/home/skbarcus/JLab/SBS/HCal/Documents/SBS_Meeting_July_2020/Event_Display_Run820_Event8_Clean.png}
  	\end{center}
  	
  	\column{0.5\textwidth}
  	\begin{center}
  		\includegraphics<1>[width=1.\linewidth]{/home/skbarcus/JLab/SBS/HCal/Documents/SBS_Meeting_July_2020/fADC_Integrals_Run820_Event8.png}
  	\end{center}
  	
  	\end{columns}

\end{frame}

\begin{frame}{Toy Example HCal Neural Network Continued}

	\begin{enumerate}
		\item[3.] Build model architecture.
		\item[4.] Train neural network.
	\end{enumerate}
	
	\begin{center}
  		\includegraphics<1>[width=0.8\linewidth]{/home/skbarcus/JLab/SBS/HCal/Documents/SBS_Meeting_July_2020/Training_Start_Clean.png}
  	\end{center}
  	
  	\begin{center}
  		\includegraphics<1>[width=0.8\linewidth]{/home/skbarcus/JLab/SBS/HCal/Documents/SBS_Meeting_July_2020/Training_End_Clean.png}
  	\end{center}

\end{frame}

\begin{frame}{Toy Example HCal Neural Network Continued}

	\begin{enumerate}
		\item[5.] Evaluate the model's performance.
		\begin{itemize}
			\item Is loss function decreasing?
			\item Is accuracy increasing?
			\item Check for overfitting.
		\end{itemize}
		\item[6.] Optimize hyperparameters.	
	\end{enumerate}
	
	  \begin{center}
  		\includegraphics<1>[width=1.\linewidth]{/home/skbarcus/JLab/SBS/HCal/Documents/SBS_Meeting_July_2020/Simple_NN_run820_50k_2layers_128_64_lr0_00001_mini64_epochs250_relu_mse.png}
  	\end{center}

\end{frame}

\begin{frame}{Other Machine Learning Applications}

	\begin{itemize}
		\item Triggers for other detectors.
		\item Data analysis:
			\begin{itemize}
				\item[--] Event classification.
				\item[--] Track reconstruction.
				\item[--] Cluster finding.
				\item[--] Regression.
			\end{itemize}
		\item Simulation:
			\begin{itemize}
				\item[--] Autoencoder/Generative Adversarial Network to more quickly produce G4SBS and other simulation data.
			\end{itemize}
		\item Unsupervised learning for event classification.
		\item Explore successful pretrained model architectures.
		\item Much more!
	\end{itemize}

\end{frame}

\begin{frame}{Summary}

	\begin{itemize}
		\item Half of detector cabled and operational with a few outstanding repairs.
		\item \setbeamercolor{alerted text}{fg=TolDarkBlue}\alert{fADCs and F1TDCs are operational}.
		\item \setbeamercolor{alerted text}{fg=mLightBrown}\alert{Cosmic tests} on one half of the HCal are under way and being analyzed.
		\item \setbeamercolor{alerted text}{fg=TolLightRed}\alert{To do}:
		\begin{itemize}
			\item[--] Finish repairing remaining broken cables on first half of HCal.
			\item[--] Cable and debug second half of HCal.
				\begin{itemize}
					\item[i] 95\%+ cables made by DSG.
					\item[ii] Installation to begin shortly.
				\end{itemize}
			\item[--] Gain match PMTs.
			\item[--] Test summing module trigger.
			\item[--] Online analysis needs development.
			\item[--] Scalers to spot check rates still need to be acquired and integrated.
		\end{itemize}
	\end{itemize}

\end{frame}

\end{document}
