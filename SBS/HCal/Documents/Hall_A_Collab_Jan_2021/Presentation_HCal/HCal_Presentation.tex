\documentclass[10pt]{beamer}

%Allow captions for subfigures
\usepackage{subcaption}

%Setup for bibtex.
\usepackage[%
  sorting=none,%
%  backend=bibtex,%
  backend=biber,%
%  style=numeric,%
  style=authoryear,% Author , Date style citations.
  %style=phys,% APS Style citations.
  pageranges=false,% Print only first page, only works with style=phys
  chaptertitle=false,% for incollections show chapter titles - true for AIP style, false for APS style
  articletitle=true,% Print article title, AIP = false, APS = true
  biblabel=brackets,% Number entries by bracked notation
  related=true,%
  isbn=false,% Don't print ISBN
  doi=false,% Don't print DOI
  url=false,% Don't print URL
  eprint=false,% Don't print eprint information
  hyperref=true,%
%  note=false,%
  firstinits=true,% Use first intials only
  maxnames=3,% Truncate after N names
  minnames=1,% Print at least N names if truncated
%  natbib=true%
 ]{biblatex}
 
%Italicize et al.
 \DefineBibliographyStrings{english}{%
  andothers = {\textit{et al}\adddot}
}

%\addbibresource{bib_Books.bib}
\addbibresource{references.bib}

%\begin{filecontents}{\jobname.bib}
%@article{Baird2002,
%author = {Baird, Kevin M and Hoffmann, Errol R and Drury, Colin G},
%journal = {Applied ergonomics},
%month = jan,
%number = {1},
%pages = {9--14},
%title = {{The effects of probe length on Fitts' law.}},
%volume = {33},
%year = {2002}
%}
%\end{filecontents}

\usetheme{metropolis}
\usepackage{appendixnumberbeamer}

\usepackage{booktabs}
\usepackage[scale=2]{ccicons}

\usepackage{pgfplots}
\usepgfplotslibrary{dateplot}

\usepackage{xspace}
\newcommand{\themename}{\textbf{\textsc{metropolis}}\xspace}

%Allows for the making of cells in tables.
\usepackage{makecell}

%\setbeamertemplate{itemize item}[circle]
%\setbeamertemplate{itemize subitem}[-]

\title{The Hadron Calorimeter for Upcoming SBS Experiments}
\subtitle{}
\date{January 22\textsuperscript{nd} 2021}
\author{The HCal Working Group \\ presented by Scott Barcus}
\institute{Jefferson Lab}
\titlegraphic{\hfill\includegraphics[height=4.0cm]{/home/skbarcus/JLab/SBS/HCal/Pictures/20190508_170610.jpg}}

\begin{document}

\maketitle

%\begin{frame}{Table of contents}
%  	\begin{columns}[T,onlytextwidth]
%  	\column{0.5\textwidth}
%
%	\vspace{10mm}
%  	\setbeamertemplate{section in toc}[sections numbered]
%  	\tableofcontents[hideallsubsections]
%  
%  	\column{0.5\textwidth}
%%		\vspace{0mm}
%%		\begin{center}
%%		\includegraphics[width=.5\linewidth]{wm_cypher_gold.png}
%%		\end{center}
%		
%		\vspace{11mm}
%		\begin{center}
%		\includegraphics[width=1.0\linewidth]{/home/skbarcus/Documents/DNP_2019/Presentation/JLab_logo_text_white1.jpg}
%		\end{center}
%		
%		\vspace{12mm}
%		\begin{center}
%		\includegraphics[width=.6\linewidth]{/home/skbarcus/Documents/DNP_2019/Presentation/DOE_Logo.jpg}
%		\end{center}
%	
%\end{columns}
%\end{frame}

\begin{frame}{HCal Overview}

	\vspace{-3mm}
    \begin{columns}[T,onlytextwidth]
  	\column{0.5\textwidth}
  	
  	\begin{center}
  	
	\begin{itemize}
		\item Design based on COMPASS HCAL1 \parencite{hcal1}.
		\item Segmented calorimeter designed to detect multiple GeV protons and neutrons.
		\begin{itemize}\itemsep0pt \parskip0pt \parsep0pt
			\item[--] 288 PMT modules (12$\times$24).
			\item[--] Four craneable subassemblies.
			\item[--] Weighs $\approx$40 tons.
			%\item[--] Wavelength shifter.
			%\item[--] Custom light guides.
			\item[--] LED fiber optics system.
		\end{itemize}
		\item SBS dipole magnet separates scattered hadrons by charge.
		\item Designed for \setbeamercolor{alerted text}{fg=TolLightRed}\alert{good time resolution} (goal 0.5 ns) and \setbeamercolor{alerted text}{fg=TolDarkBlue}\alert{good position resolution}.
		\item Energy resolution $\approx$30\%.
	\end{itemize}
	
	\end{center}
	
	\column{0.5\textwidth}
	  \begin{center}
  		\includegraphics[width=.45\linewidth]{/home/skbarcus/Documents/JLab_SS1/Seminar/HCal_External_Clean.png}
  	\end{center}
	\begin{center}
  		\includegraphics[width=1.\linewidth]{/home/skbarcus/JLab/SBS/HCal/Pictures/G4SBS/Proton_Neutron_Separation_Kin3_Clean.png}
  	\end{center}
	\end{columns}

\end{frame}

\begin{frame}{HCal Interior (288 Individual PMT Modules)}

    \begin{columns}[T,onlytextwidth]
  	\column{0.48\textwidth}
  	
  	\begin{itemize}
  		\item 40 layers of iron absorbers alternate with 40 layers of scintillator.
		\item Iron layers cause the hadrons to shower. 
		\item Scintillator layers sample the energy.
		\item Photons pass through a wavelength shifter increasing detection efficiency.
		\item Custom light guides transport photons to PMTs.
			\begin{itemize}
				\item[--] 192 12 stage 2'' Photonis XP2262 PMTs.
				\item[--] 96 8 stage 2'' Photonis XP2282 PMTs.
			\end{itemize}
  	\end{itemize}

  	\column{0.48\textwidth}
  	
  	\begin{center}
  		\vspace{5mm}
  		\includegraphics[width=1.\linewidth]{/home/skbarcus/Documents/JLab_SS1/Seminar/HCal_Interior_Clean.png}
		\vspace{10mm}
  		\includegraphics[width=1.\linewidth]{/home/skbarcus/JLab/SBS/HCal/Pictures/HCal_Single_Module_Clean.png}
  	\end{center}
  	
	\end{columns}
	
\end{frame}

\begin{frame}{$G_M^n$ Experiment}

	\begin{itemize}
		\item $G_M^n$ experiment will extract neutron magnetic form factor.
		\begin{itemize}
			\item[--] Quasielastic deuterium cross section ratios d(e,e'n)p/d(e,e'p)n \parencite{gmn_slides,gmn}.
		\end{itemize}		 
		\item HCal will detect scattered hadrons.
		\item BigBite spectrometer will detect scattered electrons.
	\end{itemize}

	\begin{center}
		\includegraphics[width=1.0\linewidth]{/home/skbarcus/JLab/SBS/HCal/Pictures/GMn_Layout_Clean_Labels.png}
	\end{center}

\end{frame}

\begin{frame}{$G_M^n$ Experiment: Science Results}

		\begin{itemize}
			\item Flavor decomposition of $G_M^n$ and $G_M^p$ allows extraction of flavor form factors.
			%\item GPD constraining sum rules ($H$ and $E$).
			\item Nucleon form factors constrain GPDs (first moments of $H$ and $E$).
			\item High $Q^2$ $G_M^n$ measurements test lattice QCD, pQCD, VMD models, and effective field theories \parencite{gmn_slides,gmn}.
		\end{itemize}

	\begin{center}
		\includegraphics[width=0.65\linewidth]{/home/skbarcus/Documents/JLab_SS1/Seminar/GMn_Results_Clean.png}
	\end{center}

\end{frame}

\begin{frame}{Geant4 Simulations}
	
	\vspace{-2mm}
	\begin{itemize}
		\item Geant4 simulations model all detectors, the target, and magnets.
			\begin{itemize}
				\item[--] Full optical photon processes (light yields and backgrounds).
			\end{itemize}
		\item Require excellent spatial resolution for high $Q^2$ SBS experiments. 
			\begin{itemize}
				\item[--] $P_N$ = 8 GeV: X (horizontal) resolution = 3.2 cm, Y (vertical) resolution = 3.8 cm.
				\item[--] $P_N$ = 2.5 GeV: X and Y resolution = 6-7 cm.
			\end{itemize}
	\end{itemize}
	
	\vspace{-2mm}
	\begin{center}
  		\includegraphics[width=0.65\linewidth]{/home/skbarcus/JLab/SBS/HCal/Documents/NIM_Paper/pictures/hcal_efficiency_resolution.png}
  	\end{center}
  	\vspace{-5mm}
	\tiny{Image image from Juan Carlos Cornejo.}
	
\end{frame}

\begin{frame}{Geant4 Simulations}

	\begin{itemize}
		\item HCal also requires nearly identical detection efficiency for protons and neutrons. 
			\begin{itemize}
				\item[--] Ratio of simulated neutron detection efficiency to proton detection efficiency. 
				\item[--] Ratio = 0.985 at 7-8 GeV. Drops to $\approx$0.966 between 2.5-4 GeV.
			\end{itemize}
	\end{itemize}
	
	\vspace{-1mm}
	\begin{center}
  		\includegraphics[width=0.65\linewidth]{/home/skbarcus/JLab/SBS/HCal/Documents/NIM_Paper/pictures/hcal_efficiency_ratio.png}
  	\end{center}
  	\vspace{-4mm}
  	\tiny{Image image from Juan Carlos Cornejo.}

\end{frame}

\begin{frame}{Data Acquisition System}

    \begin{columns}[T,onlytextwidth]
  	\column{0.55\textwidth}
  	
  	\begin{itemize}
  		\item Two VXS crates.
  		\item 18 16-channel fADC250 flash ADCs measure energy.
  			\begin{itemize}
  				\item[--] Takes numerous samples (250 MHz, 4ns).
  				\item[--] Time over threshold measurements extract timing (CFD removes time walk).
  				%\item[--] PMT traces fit by Landau.
  			\end{itemize}
  		\item 5 64-channel F1TDCs for timing.
  		\item VXS Trigger Processors (VTPs) contain FPGAs to form triggers.
  		\item \alert{Triggers:}
  			\begin{itemize}
  				\item[--] Scintillator paddle (cosmics).
  				\item[--] Summing module trigger.
  				\item[--] LED pulser trigger.
  				\item[--] BigBite coincidence trigger.
  			\end{itemize}
  	\end{itemize}
  	
  	\column{0.5\textwidth}
	\includegraphics[width=1.0\textwidth]{/home/skbarcus/JLab/SBS/HCal/Pictures/Cosmics/Landau_Fit_706_Evt1_2-10_Clean.png}
	
\vspace{10mm}	
	
	\includegraphics[width=1.0\textwidth]{/home/skbarcus/JLab/SBS/HCal/Pictures/Cosmics/Cosmic_Hit_run820_evt16_Arrow.png}

	\end{columns}

\end{frame}

\begin{frame}{fADC Pedestals}

	\begin{itemize}
		\item Typical fADC pedestal results using raw ADC units (RAU) and summed raw ADC units (SRAU).
			\begin{itemize}
				\item[--] RAU pedestal std. dev. $\approx$1 and SRAU std. dev. $\approx$30.
			\end{itemize}
	\end{itemize}

	\vspace{-6mm}

	\begin{columns}[T,onlytextwidth]
	\column{0.5\textwidth}
	\begin{center}
		\includegraphics[width=1.\linewidth]{/home/skbarcus/JLab/SBS/HCal/Pictures/Pedestals/Ped_Avg_Run1263_PMT14_Clean.png}
  	\end{center}
	
	\column{0.5\textwidth}
	\begin{center}
		\includegraphics[width=1.\linewidth]{/home/skbarcus/JLab/SBS/HCal/Pictures/Pedestals/Ped_Int_Run1263_PMT14_Clean.png}
  	\end{center}
	
	\end{columns}
	
	\begin{columns}[T,onlytextwidth]
	\column{0.5\textwidth}
	\begin{center}
		\includegraphics[width=1.\linewidth]{/home/skbarcus/JLab/SBS/HCal/Pictures/Pedestals/Avg_RAU_fADC_Ped_Run1263_Clean.png}
  	\end{center}
	
	\column{0.5\textwidth}
	\begin{center}
		\includegraphics[width=1.\linewidth]{/home/skbarcus/JLab/SBS/HCal/Pictures/Pedestals/Avg_RAU_fADC_Ped_Std_Dev_Run1263_Clean.png}
  	\end{center}
	
	\end{columns}

\end{frame}

\begin{frame}{TDC Timing Resolution}
	\vspace{-2mm}
	\begin{columns}[T,onlytextwidth]
	\column{0.6\textwidth}
	\begin{itemize}
		\item High time resolution allows separation of elastic and inelastic events.
		\item Require cosmic to be nearly \alert{`vertical'}.
			\begin{itemize}
				\item[--] Vertical F1 signals (no surrounding).
				%\item[--] No surrounding F1 signals.
			\end{itemize}
		\item TDC time:
		\begin{equation*}
			\text{T}_{\text{cor}}=\text{T}_{\text{PMT}} - \text{T}_{\text{ref}},
		\end{equation*}
		\begin{equation*}
			\text{T}_{\text{ref}}=\frac{\text{TDC\;1}+\text{TDC\;2}}{2}.
		\end{equation*}
		\item Standard deviation of single PMT:
		\begin{equation*}
    			\sigma_{PMT} = \sqrt{|\sigma_{cor}^2-\sigma_{ref}^2|}.
    		\end{equation*}
	\end{itemize}
	
	\column{0.4\textwidth}
	\begin{center}
  		\includegraphics[width=1.6\linewidth]{/home/skbarcus/JLab/SBS/HCal/Analysis/Cosmics/fADC_Timing_Res_3_12_2020/fADC_Timing_Resolution_Cuts.png}
  	\end{center}
  	\end{columns}
  	
  	\begin{center}
  		\includegraphics[width=1.\linewidth]{/home/skbarcus/JLab/SBS/HCal/Pictures/Cosmics/TDC_Timing_run820_6vert_Cropped.png}
  	\end{center}

\end{frame}

\begin{frame}{HCal LED System}

    \begin{itemize}
		\item All 288 PMT modules can observe 6 different LEDs via fiber.
		\item Each successive LED is roughly twice as bright as the previous LED.
        \item Powered by 8 LED power boxes on front-end (4 completed).
        \item Each LED power box controls 2 LED boxes on the sides of HCal (all 16 completed).
    \end{itemize}
    
    \begin{columns}[T,onlytextwidth]
	\column{0.5\textwidth}
	
	%For some incomprehensible reason these two images rotate when downloaded. To compensate they're rotated oddly in this online version.
	\vspace{-3mm}
	\begin{center}
	    \textbf{LED Power Boxes}
	    \includegraphics[width=0.8\textwidth, height=.45\textheight,angle=270]{/home/skbarcus/JLab/SBS/HCal/Documents/HCal_Update_Nov_9_2020/Pictures/LED_Power_Boxes_2.jpg}
  		%\includegraphics[width=3cm, height=4cm]{Images/LED_Power_Boxes_1.jpg}
  	\end{center}
	
	\column{0.5\textwidth}
	
	\vspace{-3mm}
	\begin{center}
		\textbf{LED Boxes}
		\includegraphics[width=0.8\textwidth, height=.45\textheight,angle=270]{/home/skbarcus/JLab/SBS/HCal/Documents/HCal_Update_Nov_9_2020/Pictures/LED_Boxes.jpg}
  	\end{center}
	
	\end{columns}

\end{frame}

%\begin{frame}{HCal LED System Cont.}
%
%    \begin{itemize}
%    		\item LEDs control boards use clock and data signals.
%		\begin{itemize}
%			\item[--] 6 bits control which LEDs are on.
%			\item[--] LED1$\rightarrow$1, LED2$\rightarrow$2, LED 3$\rightarrow$4, LED4$\rightarrow$8, LED 5$\rightarrow$16, LED 6$\rightarrow$32.
%			\item[--] Multiple LEDs: LED2 + LED4 = 10 etc. 
%		\end{itemize}
%    \end{itemize}
%
%	\vspace{-2mm}
%	\begin{center}
%	    \textbf{Clock = Blue, Data = Yellow }
%	    \includegraphics[width=1\linewidth]{/home/skbarcus/JLab/SBS/HCal/Pictures/LEDs/Good_LED_Clock_and_Data_In.jpg}
%  	\end{center}
%
%\end{frame}

\begin{frame}{LED Event Display}

	\begin{itemize}
		\item Gives a fast stable signal to work with.
		\item LED cycles programmable (i.e. turn LED1 on for 1000 triggers then LED2 for 1000).
	\end{itemize}

	\begin{center}
		\includegraphics[width=1\linewidth]{/home/skbarcus/JLab/SBS/HCal/Pictures/LEDs/LED_Pulses_Clean.png}
  	\end{center}

\end{frame}

\begin{frame}{Cosmic Ray HV Calibration}
	\vspace{-0mm}
	\begin{itemize}
		\item Calibrate the HV using cosmic rays (match fADC signals).
		\item Set cosmic ray signals so a hadron depositing the maximal energy during $G_M^n$ won't saturate electronics.
		\item Use G4SBS to determine energy deposited by hadrons and cosmics.
		\begin{itemize}
			\item[--] Plot of energy deposited in scintillators vs. incident hadron energy. (Plot thanks to Sebastian Seeds.)
		\end{itemize}
	\end{itemize}

	\vspace{-3mm}
	\begin{center}
	\includegraphics[width=0.75\linewidth]{/home/skbarcus/JLab/SBS/HCal/Documents/From_Sebastian/P_and_N_Eng_Dep_11_6_2020/Edep_Einc_HCAL_pn.pdf}
	\end{center}

\end{frame}

\begin{frame}{G4SBS for HV Calibration}

	\begin{itemize}
		\item Plot of maximum energy deposited in scintillators seen by a single PMT for $G_M^n$ max kinematic.
			\begin{itemize}
				\item[--] Max energy deposited in single PMT $\approx$ 700 MeV.
			\end{itemize}
	\end{itemize}

	\begin{center}
	\includegraphics[width=1.0\linewidth]{/home/skbarcus/JLab/SBS/HCal/Documents/Energy_Deposition/TeX_Files/Sumedep_per_Hit_13_5.png}
	\end{center}

\end{frame}

\begin{frame}{G4SBS for HV Calibration}
	\vspace{-2mm}
	\begin{itemize}
		\item Plots show average energy deposited in a single PMT's scintillators for a vertical cosmic ray. (Plots from Juan Carlos Cornejo.)
		\begin{itemize}
			\item[--] Average energy $\approx$ 14 MeV. (700 MeV/14 MeV = 50)
			\item[--] \alert{Set HV such that cosmics reach 1/50th electronics saturation.}
		\end{itemize}
	\end{itemize}

    \begin{columns}[T,onlytextwidth]
	\column{0.5\textwidth}
	
	\vspace{-6mm}
	\begin{center}
	\includegraphics[width=0.59\linewidth]{/home/skbarcus/JLab/SBS/HCal/Documents/Cosmics/Cosmic_Simulations_Cuts_JC_1-3.png}
	\end{center}
	
	\column{0.5\textwidth}
	
	\vspace{-6mm}
	\begin{center}
	\includegraphics[width=0.59\linewidth]{/home/skbarcus/JLab/SBS/HCal/Documents/Cosmics/Cosmic_Simulations_Cuts_JC_4-6.png}
	\end{center}
	
	\end{columns}

\end{frame}

%\begin{frame}{Cosmic HV Calibration}
%
%	\begin{itemize}
%		\item Maximum energy deposited by a hadron is 50 times greater than that of the average cosmic ray (700 MeV/14 MeV = 50).
%		\item 300 mV maximum off the PMTs to not saturate the amps $\rightarrow$ 1/50th of max is 6 mV off of the PMTs for an average cosmic ray.
%		\item 6 mV from the PMT is amplified to 60 mV off the amp.
%		\item 60 mV then attenuates to $\approx$30 mV at the fADCs.
%		\item 30 mV at the fADC is equal to 61 RAU.
%		\item If we set all HVs such that the average cosmic ray produces 61 RAU in the fADC then the maximum energy hadrons in $G_M^n$ will not saturate the electronics.
%	\end{itemize}
%
%\end{frame}

\begin{frame}{G4SBS Cosmics}

	\begin{itemize}
		\item Cosmic runs are currently being taken to calibrate the HV based on fADC signal.
		\item Plot shows what G4SBS predicts cosmics will look like in HCal. (Plot from Juan Carlos Cornejo.)
	\end{itemize}
	
	\begin{center}
	\includegraphics[width=1.\linewidth]{/home/skbarcus/JLab/SBS/HCal/Pictures/Cosmics/G4SBS_Cosmics.png}
	\end{center}

\end{frame}

\begin{frame}{Cosmic Results}

	\begin{itemize}
		\item Plot shows vertical (three consecutive) PMT fADC hits for HCal.
	\end{itemize}

	\begin{center}
	\includegraphics[width=1.\linewidth]{/home/skbarcus/JLab/SBS/HCal/Pictures/Cosmics/Vert_fADC_Cosmics_Run1267.png}
	\end{center}

\end{frame}

\begin{frame}{Finding Average Cosmic Amplitude}

	\begin{itemize}
		\item Cosmic hit defined as fADC signals above threshold above and below PMT module.
		\begin{itemize}
			\item[--] fADC threshold applied to central module as well to cut out pedestal.
		\end{itemize}
		\item Fit signal peak with Gaussian (skewed).
	\end{itemize}

	\begin{center}
	\includegraphics[width=1.\linewidth]{/home/skbarcus/JLab/SBS/HCal/Pictures/Cosmics/Vert_fADC_Run_1263_PMT_117_Clean.png}
	\end{center}

\end{frame}

\begin{frame}{Cosmic Calibration Progress}

	\begin{itemize}
		\item Plots display the average fADC signal (RAU) during a cosmic event versus PMT module for three runs.
			\begin{itemize}
				\item[--] Each successive run calibrates signals closer to goal of 61 RAU by adjusting HV.
			\end{itemize}
	\end{itemize}

	\begin{center}
	%\begin{figure}[!ht]
	%\begin{center}
	\begin{overprint}[12cm]
	%\begin{center}%error
	\onslide<1>\includegraphics[width=1.\linewidth]	{/home/skbarcus/JLab/SBS/HCal/Pictures/Cosmics/Avg_Cosmic_Amp_Run1263.png}
	%\end{center}%error
	%\caption[\bf{Charge Form Factors from 1352 $^3$He Fits with no $\chi^2_{max}$ cut}]{
	%{\bf{Charge Form Factors from 1352 $^3$He Fits with no $\chi^2_{max}$ cut.}} }
	%\label{fig:3he_fch_no_cut}
	\onslide<2>\includegraphics[width=1.\linewidth]	{/home/skbarcus/JLab/SBS/HCal/Pictures/Cosmics/Avg_Cosmic_Amp_Run1265.png}
	%\caption[\bf{Charge Form Factors from 852 $^3$He Fits surviving a $\chi^2_{max}$ = 500 cut}]{
	%{\bf{Charge Form Factors from 852 $^3$He Fits surviving a $\chi^2_{max}$ = 500 cut.}} }
	%\label{fig:3he_fch_cut}
	%\end{center}%error
	\onslide<3>\includegraphics[width=1.\linewidth]	{/home/skbarcus/JLab/SBS/HCal/Pictures/Cosmics/Avg_Cosmic_Amp_Run1267.png}
	\end{overprint}
	%\end{center}
	%\end{figure}
	\end{center}


\end{frame}

\begin{frame}{Machine Learning Detector Trigger for HCal}

	\begin{itemize}
		\item \alert{Motivation:}
			\begin{itemize}
				\item[--] High background rates obscure physics signals.
			\end{itemize}
		\item \alert{Traditional Solutions:}
			\begin{itemize}
				\item[--] Energy threshold cuts.
				\item[--] Prescaling the data.
				\item[--] Decreasing the beam current.
			\end{itemize}
		\item \alert{Machine Learning Solution:}
			\begin{itemize}
				\item[--] Train a neural network to classify detector events (e.g. p, n, $\pi$).
				\item[--] Use data from G4SBS converted to detector output to train NN.
				\item[--] Load trained NN onto VTP FPGA (fast) to use as HCal trigger.
			\end{itemize}
		\item \alert{Goal:}
			\begin{itemize}
				\item[--] Demonstrate that NNs can be loaded onto VTPs for triggering JLab detectors.
				\item[--] Allow HCal to run at higher current with a cleaner trigger.
			\end{itemize}
	\end{itemize}

\end{frame}

\begin{frame}{Proposed Convolutional Neural Network Architecture}

	\begin{center}
  		\includegraphics<1>[width=0.85\linewidth]{/home/skbarcus/JLab/SBS/HCal/Machine_Learning/Pictures/HCal_Generic_CNN.png}\\
  		%\tiny{Image from https://www.mdpi.com/2076-3417/9/21/4500.}
  	\end{center}

	\vspace{-3mm}
	\begin{itemize}
		%\item Started with image classification. 
		\item \alert{PMT pulse shapes are essentially images.}
			\begin{itemize}
				\item[--] Each event every PMT has several fADC samples and a TDC value.
			\end{itemize}
		%\item Fully connected NNs assume each neuron connects to each neuron in the next layer and that each connection is equally important.
%			\begin{itemize}
%				\item[--] \setbeamercolor{alerted text}{fg=TolLightRed}\alert{The location of PMTs relative to one another matters!}
%			\end{itemize}
		\item CNN scans across the image with a kernel creating \setbeamercolor{alerted text}{fg=TolDarkBlue}\alert{filters} which \alert{identify localized features} (like hits).
		\begin{itemize}
			\item[--] \setbeamercolor{alerted text}{fg=TolLightRed}\alert{Detector geometry preserved.}
			\item[--] Pooling decreases dimensionality by merging adjacent signals.
			\item[--] Batch normalization improves speed and regularization.
			\item[--] Dropout helps reduce overfitting.
		\end{itemize}
	\end{itemize}

\end{frame}

\begin{frame}{Toy Example HCal Neural Network}

	\begin{itemize}
		\item Create CNN to identify events with large cosmic signals. 
		\begin{itemize}
			\item[--] Input: PMT fADC integrals.
			\item [--] Note: Traditional methods work better than this illustrative example.
		\end{itemize}			
		\item \alert{Tools:} ROOT, Python, Numpy, Scikit-learn, Tensorflow, Keras, Google Colaboratory (GPUs).
		\item \setbeamercolor{alerted text}{fg=mLightBrown}\alert{99\% accuracy} with small amount of optimizing!
	\end{itemize}
	
	\begin{center}
  		\includegraphics<1>[width=1.\linewidth]{/home/skbarcus/Documents/ML_LDRD/LDRD_Proposal_Talk/CNN_Results_Clean.png}
  	\end{center}

\end{frame}

\begin{frame}{HCAL-J Summary}

	\begin{itemize}
		\item Detects protons and neutrons for future JLab Hall A SBS experiments.
			\begin{itemize}
				\item[--] Measure nucleon form factors up to high $Q^2$.
			\end{itemize}
		\item Energy resolution $\approx$30\%.
		\item \setbeamercolor{alerted text}{fg=TolLightRed}\alert{High time resolution ($\approx$0.5 ns).}
		\item \setbeamercolor{alerted text}{fg=TolDarkBlue}\alert{Excellent position resolution (as low as 3-4 cm).}
		\item Similar detection efficiency for protons and neutrons.
		\item LED system for calibrations (currently studying HV vs. gain).
		\item Cosmic ray testing and calibrations in progress.
		\item ML particle ID detector trigger research ongoing. 
			\begin{itemize}
				\item[--] Serve as test bed for FPGA based ML detector triggers at JLab and hopefully create a cleaner HCal trigger.
			\end{itemize}
	\end{itemize}

\end{frame}

\begin{frame}{Acknowledgments}
Thanks to \alert{Gregg Franklin} for his many dedicated years designing and overseeing the construction of HCal. Thanks to \alert{Universit\'{a} di Catania} for their major financial contributions. Many other people and institutions were involved in making HCal possible, including, but not limited to:
    \begin{itemize}
        \item Thanks to the many students who have worked on HCal including \setbeamercolor{alerted text}{fg=TolDarkBlue}\alert{Alexis Ortega}, \alert{So Young Jeon}, \alert{Jorge Pe\~{n}a}, \alert{Carly Wever}, and  \alert{Dimitrii Nikolaev}. 
        \item Thanks to our current students \setbeamercolor{alerted text}{fg=TolLightRed}\alert{Vanessa Brio} and \alert{Sebastian Seeds}.
        \item Thanks to \setbeamercolor{alerted text}{fg=mLightBrown}\alert{Alexandre Camsonne} for helping us get the DAQ working.
        \item Thanks to \alert{Chuck Long} for all his help fixing and acquiring things.
        \item Thanks to \alert{Bryan Moffit} for DAQ help.
        \item Thanks also to \alert{Brian Quinn} and \alert{Bogdan Wojtsekhowski}.
        \item Thanks to \setbeamercolor{alerted text}{fg=TolDarkBlue}\alert{Vanessa Brio}, \alert{Cattia Petta}, and \alert{Vincenzo Bellini} for their cosmic commissioning efforts.% last summer.
        %\item Finally welcome to \setbeamercolor{alerted text}{fg=TolLightRed}\alert{Dimitrii Nikolaev}, \alert{Jim Napolitano}, \alert{Donald Jones}, and \alert{Kent Paschke}.
    \end{itemize}

\end{frame}

\begin{frame}{Questions?}%{References}
	
	\setbeamertemplate{bibliography item}{}%Removes page icon in from of each references.
	\renewcommand*{\bibfont}{\scriptsize}%Change bib font size.
	%\printbibliography[heading=bibintoc, title=References]
	\printbibliography
	%\bibliographystyle{plainnat}
	%\bibliography{bib_Books}
	
\end{frame}

\begin{frame}{LED High Voltage Calibration}

	\begin{itemize}
		\item Want to know fADC signal to HV calibration (gain calibration).
		\item Program LEDs to cycle during run in 1000 event increments.
			\begin{itemize}
				\item[--] Off$\rightarrow$LED1$\rightarrow$LED2$\rightarrow$LED3$\rightarrow$LED4$\rightarrow$LED5$\rightarrow$Repeat
			\end{itemize}
		\item Data was taken on the left half of HCal at eight voltages.
			\begin{itemize}
				\item[--] -1200 V, -1250 V, -1300 V, -1350 V, -1375 V, -1400 V, -1425 V, -1450 V.
				\item[--] Five full cycles of data were taken (5k events/setting).
			\end{itemize}
	\end{itemize}
	
	\begin{columns}[T,onlytextwidth]
	\column{0.8\textwidth}
	\begin{itemize}
		\item At low voltages and dimmer LEDs sometimes there is no PMT response.
		\item At high voltage and brighter LEDs sometimes the fADC saturates.
	\end{itemize}
	
	\column{0.2\textwidth}
	\vspace{-4mm}
	\begin{center}
		\includegraphics[width=1.\linewidth]{/home/skbarcus/JLab/SBS/HCal/Pictures/LEDs/Single_Saturated_fADC.png}
  	\end{center}
  	
	\end{columns}

\end{frame}

\begin{frame}{SRAU for 5 LEDs}

	\begin{itemize}
		\item Fit each LED fADC signal histogram for each HV setting.
		\item Pictured below PMT 129 at a single voltage looking at 5 different LEDs.
		\begin{itemize}
			\item[--] Each LED roughly doubles in brightness as expected.
			\item[--] Plot is not pedestal subtracted.
		\end{itemize}
	\end{itemize}

	\begin{center}
		\includegraphics[width=1.\linewidth]{/home/skbarcus/JLab/SBS/HCal/Pictures/LEDs/5_LED_Peaks_Clean.png}
  	\end{center}

\end{frame}

\begin{frame}{Selected Results}

	\begin{itemize}
		\item The CMU PMTs have exponents $\approx$ 10-11.
	\end{itemize}
	
	\begin{center}
		\includegraphics[width=1.\linewidth]{/home/skbarcus/JLab/SBS/HCal/Pictures/LEDs/CMU_PMT_Good_Chi2_Clean.png}
  	\end{center}

\end{frame}

\begin{frame}{Selected Results}

	\begin{itemize}
		\item The JLab PMTs have exponents $\approx$ 8.
	\end{itemize}
	
	\begin{center}
		\includegraphics[width=1.\linewidth]{/home/skbarcus/JLab/SBS/HCal/Pictures/LEDs/JLab_PMT_Good_Chi2_Clean.png}
  	\end{center}

\end{frame}

\end{document}
