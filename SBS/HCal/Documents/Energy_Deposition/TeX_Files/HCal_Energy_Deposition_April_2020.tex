\documentclass[10pt]{beamer}

%Allow captions for subfigures
\usepackage{subcaption}

%Setup for bibtex.
\usepackage[%
  sorting=none,%
%  backend=bibtex,%
  backend=biber,%
%  style=numeric,%
  style=authoryear,% Author , Date style citations.
  %style=phys,% APS Style citations.
  pageranges=false,% Print only first page, only works with style=phys
  chaptertitle=false,% for incollections show chapter titles - true for AIP style, false for APS style
  articletitle=true,% Print article title, AIP = false, APS = true
  biblabel=brackets,% Number entries by bracked notation
  related=true,%
  isbn=false,% Don't print ISBN
  doi=false,% Don't print DOI
  url=false,% Don't print URL
  eprint=false,% Don't print eprint information
  hyperref=true,%
%  note=false,%
  firstinits=true,% Use first intials only
  maxnames=3,% Truncate after N names
  minnames=1,% Print at least N names if truncated
%  natbib=true%
 ]{biblatex}
 
%Italicize et al.
 \DefineBibliographyStrings{english}{%
  andothers = {\textit{et al}\adddot}
}

\addbibresource{references.bib}

\usetheme{metropolis}
\usepackage{appendixnumberbeamer}

\usepackage{booktabs}
\usepackage[scale=2]{ccicons}

\usepackage{pgfplots}
\usepgfplotslibrary{dateplot}

\usepackage{xspace}
\newcommand{\themename}{\textbf{\textsc{metropolis}}\xspace}

%Allows for the making of cells in tables.
\usepackage{makecell}

%Allows for use of long table that can go over multiple slides.
\usepackage{longtable}

%\setbeamertemplate{itemize item}[circle]
%\setbeamertemplate{itemize subitem}[-]

\title{HCal Energy Deposition Study}
\subtitle{}
\date{April 2\textsuperscript{st} 2020}
\author{Scott Barcus}
\institute{Jefferson Lab}
% \titlegraphic{\hfill\includegraphics[height=1.5cm]{logo.pdf}}

\begin{document}

\maketitle

\begin{frame}{Current Status}

	\begin{table}[t]
	\centering
	%\caption{Spectrometer Central Kinematics}%Prints title above table.
	\begin{tabular}{|c|ccccccc|}
	\hline
	\makecell{Kine\\$[$GeV$^2]$} & \makecell{HCal\\Evts} & \makecell{Tot.\\Hits} & \makecell{Max Edep\\PMT\\ $[$row-col$]$} & \makecell{Max\\Edep\\PMT\\$[$MeV$]$} & \makecell{Max\\Edep\\PMT\\NPE} & \makecell{Max\\Edep\\All\\PMTs\\$[$MeV$]$} & \makecell{Max\\Edep\\All\\PMTs\\NPE}\\
	\hline
	3.5 & 5082 & 11747 & r13c5 & 267 & 1469 & 407 & 2236 \\
    4.5 & 6070 & 18859 & r12c4 & 308 & 1694 & 465 & 2559 \\
    5.7 & 7690 & 30942 & r11c1 & 418 & 2302 & 493 & 2712 \\
    8.1 & 7874 & 38803 & r14c3 & 497 & 2733 & 726 & 3893 \\
    10.2 & 7539 & 41205 & r10c3 & 570 & 3132 & 797 & 4383 \\
    12.0 & 8781 & 53540 & r18c7 & 669 & 3682 & 960 & 5280 \\
    13.5 & 6711 & 34567 & r10c0 & 740 & 4070 & 977 & 5373 \\
	\hline
	\end{tabular}
	\label{tab:status}
	%\caption*{Current Status}
	%\caption*{For the high rate kinematics, these measurements will not be statistics limited and those points 		will be used to check systematics. For the lowest Q$^2$, this measurement can be used to precisely determine 	the product of beam-target polarization.} %Caption* supresses printing of second caption saying Table number again.
	\end{table}
	
\end{frame}



\end{document}
